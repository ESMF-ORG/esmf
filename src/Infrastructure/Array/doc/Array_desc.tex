% $Id$
%
% Earth System Modeling Framework
% Copyright (c) 2002-2022, University Corporation for Atmospheric Research,
% Massachusetts Institute of Technology, Geophysical Fluid Dynamics
% Laboratory, University of Michigan, National Centers for Environmental
% Prediction, Los Alamos National Laboratory, Argonne National Laboratory,
% NASA Goddard Space Flight Center.
% Licensed under the University of Illinois-NCSA License.

The Array class is an alternative to the Field class for representing
distributed, structured data.  Unlike Fields, which are built to carry
grid coordinate information, Arrays only carry information about the
{\it indices} associated with grid cells.  Since they do not have coordinate
information, Arrays cannot be used to calculate interpolation weights.
However, if the user supplies interpolation weights, the Array sparse
matrix multiply (SMM) operation can be used to apply the weights and transfer
data to the new grid.  Arrays carry enough information to perform
redistribution, scatter, and gather communication operations.

Like Fields, Arrays can be added to a State and used in inter-Component
data communications.  Arrays can also be grouped together into ArrayBundles,
allowing operations to be performed collectively on the whole group.  One
motivation for this is convenience; another is the ability to schedule
optimized, collective data transfers.

From a technical standpoint, the ESMF Array class is an index space based,
distributed data storage class. Its purpose is to hold distributed user data.
Each decompositon element (DE) is associated with its own memory allocation. The
index space relationship between DEs is described by the ESMF DistGrid class.
DEs, and their associated memory allocation, are pinned either to a specific
perisistent execution thread (PET), virtual address space (VAS), or a single
system image (SSI). This aspect is managed by the ESMF DELayout class. Pinning
to PET is the most common mode and is the default.

The Array class offers common communication patterns within the index space
formalism. All RouteHandle based communication methods of the Field,
FieldBundle, and ArrayBundle layers are implemented via the Array SMM operation.
