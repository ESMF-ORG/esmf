\subsection{User Support}
\label{sec:usr_support}

\subsubsection{Roles}
The {\bf Advocate} is the staff person assigned to a particular code e.g. GEOS-5. See section ~\ref{core} for a full definition and list of responsibilities. 
The {\bf Handler} is the staff person assigned to solve a support ticket. The Advocate and the Handler may be the same person or they may be different. See section ~\ref{core} for complete definition and list of responsibilities.

\subsubsection{Support Categories}
{\bf New} is a request that has not been replied to.
{\bf Closed} is a request that has been fixed to the user's satisfaction.
{\bf Pending} is a request that has been fixed to the Handler's satisfaction but has not yet been approved by the user.

\subsubsection{Summary Work Flow}
\begin{enumerate}
\item Message received.
\item The Integrator or in his absence the Support Lead, generates a GitHub issue.
\item If the request contains more than one topic, then Integrator will open multiple tickets, one per topic. This can been done initially if obvious, or later if more research indicates it is necessary. 
   \begin{itemize}
   \item The top line of the entry should be WHO: <Requester Name>.
   \item Indicate the institution and model if known.
   \item Keep title of initial email and the title of the SF ticket the
same or close enough to be able to determine they are one and the same.
   \item Assign the ticket to the staff person best able to solve the ticket's issue. 
   \end{itemize}

\item Initial contact is made by:

\begin{itemize}
\item The Handler assigned by the Integrator in the ticket. 
\item The Support Lead if the Handler will be unavailable for more than a week. 
\end{itemize}

\item The Handler works to solve the tickets issues. He or she will communicate 
periodically with the ticket's originator and will keep the rest of the Core team 
informed on the tickets progress at the monthly ticket review meetings.  Once the 
issue has been solved, the ticket will be marked pending by the Handler.  

\item At this point, the Handler contacts the originator to gauge their satisfaction with
the solution. If the originator is satisfied, the ticket may be closed, and the mail 
folder on the IMAP server moved from Open to Closed by the Support Lead.  If the customer
does not respond, an attempt at contact will be made once a month for two months. 
If after this period, the originator still does not reply, a pending ticket may be closed
with final notification to the originator.   
\end{enumerate}

\subsubsection{General Guidelines for Handling Tickets}
\begin{itemize}
\item Include title and ticket number on all correspondence.
\item Make initial contact within 48 hours even if just to say message received.
\item The email address for ticket originators can be found in either freeCRM or the mail archive. Do not hesitate to contact the Support Lead if a required email address can not be found. 
\item Copy esmf\_support@ucar.edu on all replies.
\item Bugs that are fixed should be marked Closed, and Fixed. They should never be deleted. 
\item Bugs that are duplicates should be marked Deleted, and Duplicate. 
\item If the main issue in a Bug, Feature Request, or Support Request has not been implemented it should stay Open.
\item Users are always notified via email when their ticket is being closed even if they have been unresponsive.  
\item If the solution to a ticket involves a test code, this should be incorporated into the code body as standard test. It should not be sent to the user as an unofficial code fragment. 
\item If the solution to a ticket involved changes to the code, the user should be given a stable beta snapshot. The user should not be directed to the HEAD, which is inherently unstable.  
\item If a ticket involves an older version of the code and a computing environment that the current distribution runs on, the ticket should be considered for closure when there is no means of testing or fixing the older code. 
\item The Handler is responsible for changing the status of tickets assigned to them.  
\end{itemize}

\subsubsection{esmf\_support@ucar.edu Mail Archives}
The Support Lead manages the archive of esmf\_support@ucar.edu email traffic and is responsible for the creation of ticket folders, component folders, and the proper placement of mail messages. The archive is located on the main CISL IMAP server and can be accessed by any Core member.  Contact the Support Lead if you wish your local mail client enabled to view the archive.  The IMAP archive will have the following appearance:
\begin{itemize}
\item{Component Name}
  \begin{itemize}
    \item{Open}
      \begin{itemize}
        \item{Numbered Ticket Folder}
        \item{Numbered Ticket Folder}
        \item{Numbered Ticket Folder}
      \end{itemize}
    \item{Closed}
      \begin{itemize}
        \item{Numbered Ticket Folder}
        \item{Numbered Ticket Folder}
        \item{Numbered Ticket Folder}
      \end{itemize}
    \end{itemize}
\item{Component Name}
\end{itemize}

The following rules apply to the above:

\begin{itemize}
\item Email messages will be filed by component and number. 
\item A folder labeled with the request number will be created. 
\item This folder will then be placed in the components Open folder until closed.  
\item The Support Lead will copy each related email message to its numbered folder. 
\item When a ticket has been closed, the Support Lead will move the numbered folder from the components Open folder to its Closed folder.
\item There will be only one New folder to which highly active tickets may be placed 
for easier filing at the discretion of the Support Lead. 
\end{itemize}

\subsubsection{INFO:Code (subject) mail messages}
\label{infomail}

Advocates need to share the information they have received from their codes with the rest 
of the Core team. This will be done by sending an email to esmf\_support@ucar.edu with a 
subject line labeled INFO: Code e.g. INFO: CCSM, INFO: GEOS-5. These messages will be 
filed on the IMAP server (see above section) under the code referenced. All information 
about a code that is general and not related to a specific support request will be archived 
in this manner. 

\subsubsection{freeCRM}
A client relationship management tool (freeCRM http://www.freecrm.com) is being used 
to archive codes, their affiliated contacts, degree of componentization, issues, and 
applicable funding information if known. The following is a list of roles and 
responsibilities associated with this software:
\begin{itemize}
\item Advocates are responsible for the accuracy and completeness of all information 
associated with codes to which they are assigned.  This information includes a pull 
down menu that specifies the state of the code's ESMF'ization. This piece of 
information is critical and needs to be updated whenever an Advocate updates his or 
her codes. Other information includes type of code, parent agency etc. This 
information will be reviewed on a semi-annual basis. 
\item The Integrator is responsible for creating a back up of all freeCRM data on a 
monthly basis.
\item The Core Team Manager is responsible for the accuracy and completeness of all 
funding related information.
\item The Support Lead is responsible for creating code 'companies' and informing 
the Integrator of any additions so that the back up scripts can be modified. He or 
she is additionally responsible for conducting semi-annual quality control checks of 
all information in the system. 
\item All team members are responsible for updating and adding to the list of 
contacts. 
\end{itemize} 

\subsubsection{Annual Code Contact}
Once a year all codes in the freeCRM data base will be contacted in order to gauge 
their development progress, and to update our component metrics. This process will 
contain the following steps:
\begin{itemize}
\item Advocates will login into freeCRM and get a list of all their codes. 
This list will be emailed to esmf\_core@cgd.ucar.edu.
\item Advocates will review their list and determine which codes on the list need to be
contacted. Contact is not needed if sufficient knowledge is already known about a 
code.
\item Advocates will review all the information contained in freeCRM concerning their 
assigned codes AND review all the esmf\_support traffic for the last year.  
\item Advocates will draft the contact email and send it to esmf\_core@cgd.ucar.edu to be 
reviewed by the Core Team Manager.  Once corrected, the Advocates may send their email. Since 
this is a group level effort, the email message may be signed ``The ESMF Team'' if 
desired. 
\item The Support Lead will track the draft and completed emails as well as the 
responses and will provide a report to the Core Team Manager at the end of the 
process.  
\item As responses come in, the Advocates are responsible for updating the 
information in freeCRM.  
\item The Support Lead will tally the results and update the components page on the
ESMF Web site and will also update the components metric chart.  
\end{itemize}


\subsubsection{Dealing with Applications that use ESMF}
More and more applications are being distributed with embedded ESMF interfaces. It may 
difficult to determine if a reported problem with one of these applications is related 
to an incorrect ESMF implementation, a true ESMF bug, or an issue within the parent model.
The following are several definitions:
\begin{itemize}
\item End User: A person who downloads or otherwise receives an application that contains 
ESMF code.  While they may be trying to modify this application, they were not the person 
or persons who originally inserted ESMF into the application. Most likely, they will be 
entirely unfamiliar with ESMF.
\item Application Developer: The person or persons who took a model, inserted ESMF code, 
and made the resulting application available to others.
\end{itemize} 
The following are some guidelines for dealing with such Applications that use ESMF:
\begin{itemize}
\item For support requests related to applications that include ESMF,  our primary contact 
for resolving the request should be the developers of the distributed application and not 
the End User.  As such, every effort should be made to identify and contact the developers 
of the distributed application in order to make them aware of the reported issue and to get 
them actively engaged in resolution of the problem. Additionally, they should be cc'ed on 
all correspondence with the End User.
\item During the resolution of the issue, it will be necessary to cc all email traffic to the 
End User.  In dealings with the End User emphasize that the ESMF group is committed to any user 
of ESMF regardless of source. That commitment is predicated, however, on participation of the 
application developers.
\item The Handler should establish which version of ESMF the application is using.   
\item The Handler should try and determine whether the ESMF code in question was modified in any 
way by the Application Developers. 
\item The Handler should try and determine whether the code in question has ESMF interface names 
but is not ESMF code. The time manager in WRF falls into this category. It has ESMF interfaces 
but was not developed by us.  
\item It will be solely the Core Team's discretion whether or not to support older versions of ESMF, 
ESMF code that has been modified by others, or code that uses ESMF interface names but was developed 
entirely separate from ESMF.
\item In no way should the Handler try running the End User's code.
\item In the event that the developers of the distributed application are 
unknown, unreachable, or uncooperative, the End User must be politely informed that the group 
can not troubleshoot code belonging to another group. This will have to been handled with a 
degree of sensitivity because it is likely that the end user has already tried to contact the 
application developers without success. 
\end{itemize}

