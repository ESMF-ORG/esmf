%  $Id$

% there is a separate copy of this under build/doc which is
% now used only by the implementation report.  this was separated
% out because it replicated information which was in the ESMF_install and
% ESMF_quickstart files.

\subsection{The ESMF Build System}
\label{sec:make}
For most users the description of the build system in previous
sections should be sufficient.  Some users, however, may wish to have
a more detailed knowledge of the make system either for configuring
different build options or for porting to unsupported platforms.

\subsubsection{General structure}

The main components of the build system are:

\label{sec:BuildOptions}
\begin{itemize}
\item{{\bf Build directories with makefile fragments}}

There are two directories containing makefile fragment files used by
the ESMF build system.  

The {\tt build} directory contains the generic makefile fragment file
{\tt common.mk} that is included by the top level {\tt makefile} in the source
tree. The {\tt common.mk} contains generic build system settings and build
rules used across all platforms.  A user should have no reason to edit
{\tt common.mk}.

The {\tt build\_config} directory contains subdirectories with makefile
fragments ({\tt build\_rules.mk}) for each supported platform defining
compilers, compiler flags and the various other definitions that are
necessary to build on each platform. One of the {\tt build\_rules.mk} files
will be included by the {\tt build/common.mk} file depending on the values of
the environment variables ESMF\_OS, ESMF\_COMPILER and ESMF\_SITE. See below
for more details on environment variables.

\item{{\bf Environment variables}}

Environment variables with the prefix {\tt ESMF\_} are used to pass user
specified information to the ESMF build system. A full list of {\tt ESMF\_}
environment variables is provided in section \ref{EnvironmentVariables} of this
document.

Most environment variables are optional and the ESMF build system will use
default settings if it finds these variable unset. One piece of information that
must always be provided by setting the respective environment variable is the
root of the ESMF directory. There are three sets of source codes the build
system supports. All need environment variables set to point to their top 
level source code directories.

\begin{description}

\item{ESMF Library} 

To build the ESMF library, ESMF\_DIR needs to be set to the top level ESMF
library source code directory.

\item{Implementation Report} 

The build system needs ESMF\_IMPL\_DIR set to the top level source
code directory of the Implementation Report source tree to build the
report and to build and run the examples.

\item{EVA Applications} 

An EVA source code tree does not contain a copy of the ESMF build
system.  Instead it uses a copy found in an ESMF library source code
tree.  Building the EVA applications requires that ESMF\_EVA\_DIR and
ESMF\_DIR be set.  ESMF\_EVA\_DIR has to be set to the top directory
of the EVA source code.  ESMF\_DIR has to be set to the top directory
of an ESMF source code tree.

\end{description}


\item{{\bf Makefiles}}

Every source tree contains a {\tt makefile} in its top level directory. This
{\tt makefile} includes the {\tt common.mk} file from the {\tt build} directory
which in turn includes the platform specific {\tt build\_rules.mk} file from
one of the {\tt build\_config} subdirectories. The top level {\tt makefile}
contains makefile settings specific for the source code that it is found in.

Each directory in the source tree contains a {\tt makefile} which includes
the top level {\tt makefile}. These local makefiles include definitions that
allow the local files and documents to be built.
\end{itemize}

\subsubsection{Build configuration}

A single makefile or makefile fragment from the build system never
constitutes a complete set of build rules and settings.  Starting from
the local makefile, successive include commands are used to string
together makefiles and makefile fragments to create a complete system
of build rules and settings.  Configuration of the build system is
done by including a configuration makefile fragment. A configuration for a
specific machine or compiler is referred to as a site configuration.

The string of files included is fairly short.  Makefiles below the top
level makefile include the top level makefile. The top level makefile
includes {\tt build/common.mk} and then {\tt build/common.mk} includes a
configuration file from the {\tt build\_config} directory.  The configuration
files in the {\tt build\_config} directory contain the platform and site
specific build settings.  The os, compiler and site that a file
configures is determined by its name.  The configuration makefile
fragments follow the naming convention

\begin{verbatim}
    build_config/ESMF_OS.ESMF_COMPILER.ESMF_SITE/build_rules.mk
\end{verbatim}

where {\tt ESMF\_OS}, {\tt ESMF\_COMPILER} and {\tt ESMF\_SITE} are environment
variables either set by the user or given default values by the build
system. {\tt ESMF\_OS} is the target operating system. If the build is performed
{\em on} the target system {\tt ESMF\_OS} will typically have the value
returned by the command {\tt uname -s}. {\tt ESMF\_COMPILER} is the compiler
name. {\tt ESMF\_SITE}, if set, is generally the current machine name, the
location, or the organization (e.g. mit, cola).  If there are no site specific
files for a particular platform, then {\tt ESMF\_COMPILER} and {\tt ESMF\_SITE}
will be set to {\tt default}.  Examples:

\begin{verbatim}
    ! Default configuration for IBM AIX systems
    build_config/AIX.default.default/build_rules.mk
    
    ! Linux configuration using lahey compilers.    
    build_config/Linux.lahey.default/build_rules.mk
\end{verbatim}

\subsubsection{Source code configuration}

Some of the ESMF C++ and Fortran source files contain preprocessor directives
to configure the source code for specific platforms.  The directives are 
included in the source code and are pre-processed before the source code is 
compiled.  The directives are used to determine among other things, the size 
of variable types.

The ESMF build system provides preprocessor directives in 
{\tt ESMC\_Conf.h} and {\tt ESMF\_Conf.inc} files
that are included in the source code. These files are located in

\begin{verbatim}
    build_config/ESMF_OS.ESMF_COMPILER.ESMF_SITE/ESMC_Conf.h
    build_config/ESMF_OS.ESMF_COMPILER.ESMF_SITE/ESMF_Conf.inc
\end{verbatim}

where {\tt ESMF\_OS}, {\tt ESMF\_COMPILER} and {\tt ESMF\_SITE} are
environment variables set by the user or given default values be the
build system.  Based on the settings of these environment variables
the build system provides a path to the correct files during
source code compilation.

\subsection{Porting ESMF to New Platforms}

The ESMF build system can be ported to other Unix platforms by adding a new
platform specific makefile fragment and two associated configuration files.
These files ({\tt build\_rules.mk}, {\tt ESMC\_Conf.h}, {\tt ESMF\_Conf.inc})
must be placed into a new subdirectory of the {\tt build\_config} directory,
following the {\tt ESMF\_OS.ESMF\_COMPILER.ESMF\_SITE} naming convention.

When porting to a new platform it is often helpful to start with a copy 
of the configuration of an existing ESMF port. You may, for example, want to
start with a copy of the {\tt build\_config/Linux.g95.default} directory when
working on a new Linux configuration.

\subsubsection{Customizing the {\tt build\_rules.mk} fragment}

The purpose of the {\tt build\_rules.mk} makefile fragment is to customize the
build procedure for a specific platform. The customization is done via makefile
variables. The main {\tt makefile} at the top level of the ESMF directory
structure first includes the {\tt common.mk} makefile fragment. This common
makefile fragment defines a large number of variables, setting them either to
generally valid default values or to specific values the user has set in their
environment using {\tt ESMF\_} style environment variables.

The platform specific {\tt build\_rules.mk} makefile fragment is included by
{\tt common.mk} {\em after} the variables have been initialized, but 
{\em before} any rules are defined in {\tt common.mk} using these variables.
This gives {\tt build\_rules.mk} a chance to modify these variables as it may
be necessary to accommodate platform specific properties.

Fortunately only a very small subset of variables pre-defined in {\tt common.mk}
typically need to be modified or overridden in {\tt build\_rules.mk} with 
platform specific settings. However, there are some variables that {\em must}
be set in every {\tt build\_rules.mk} file. These are variables that are not
pre-set in {\tt common.mk}.

\begin{description}

\item[ESMF\_CXXDEFAULT]
Default C++ compiler to be used on this platform. This variable will be used
by {\tt common.mk} to set the associated {\tt ESMF\_CXX} variables.
\item[ESMF\_CXXCOMPILER\_VERSION]
Command that when executed will provide information about the version of the
C++ compiler to stdout.
\item[ESMF\_F90DEFAULT]
Default Fortran compiler to be used on this platform. This variable will be used
by {\tt common.mk} to set the associated {\tt ESMF\_F90} variables.
\item[ESMF\_F90COMPILER\_VERSION]
Command that when executed will provide information about the version of the
F90 compiler to stdout.
\item[ESMF\_MPIRUNDEFAULT]
Default MPI job launch facility to be used on this platform. This variable will
be used by {\tt common.mk} to set the associated {\tt ESMF\_MPIRUN} variables.

\end{description}

The following is a complete alphabetical list of variables that are pre-set 
in {\tt common.mk} before {\tt build\_rules.mk} is included. Some of these
variables correspond to {\tt ESMF\_} environment variables while others have 
a more complicated dependency on the environment variables set by the user.

\begin{description}

\item[ESMF\_ABI]
\item[ESMF\_APPSDIR]
\item[ESMF\_AR]
\item[ESMF\_ARCREATEFLAGS]
\item[ESMF\_ARCREATEFLAGSDEFAULT]
\item[ESMF\_ARDEFAULT]
\item[ESMF\_AREXTRACTFLAGS]
\item[ESMF\_AREXTRACTFLAGSDEFAULT]
\item[ESMF\_ARRAY\_LITE]
\item[ESMF\_BOPT]
\item[ESMF\_BUILD]
\item[ESMF\_BUILD\_DOCDIR]
\item[ESMF\_COMM]
\item[ESMF\_COMPILER]
\item[ESMF\_CONFDIR]
\item[ESMF\_CPP]
\item[ESMF\_CPPDEFAULT]
\item[ESMF\_CXXCOMPILECPPFLAGS]
\item[ESMF\_CXXCOMPILEOPTS]
\item[ESMF\_CXXCOMPILEPATHS]
\item[ESMF\_CXXCOMPILEPATHSLOCAL]
\item[ESMF\_CXXCOMPILER]
\item[ESMF\_CXXCOMPILERDEFAULT]
\item[ESMF\_CXXESMFLINKLIBS]
\item[ESMF\_CXXLINKER]
\item[ESMF\_CXXLINKERDEFAULT]
\item[ESMF\_CXXLINKLIBS]
\item[ESMF\_CXXLINKOPTS]
\item[ESMF\_CXXLINKPATHS]
\item[ESMF\_CXXLINKRPATHS]
\item[ESMF\_CXXOPTFLAG]
\item[ESMF\_CXXOPTFLAG\_G]
\item[ESMF\_CXXOPTFLAG\_O]
\item[ESMF\_CXXOPTFLAG\_X]
\item[ESMF\_DIR]
\item[ESMF\_DOCDIR]
\item[ESMF\_EXDIR]
\item[ESMF\_F90COMPILECPPFLAGS]
\item[ESMF\_F90COMPILEFIXCPP]
\item[ESMF\_F90COMPILEFIXNOCPP]
\item[ESMF\_F90COMPILEFREECPP]
\item[ESMF\_F90COMPILEFREENOCPP]
\item[ESMF\_F90COMPILEOPTS]
\item[ESMF\_F90COMPILEPATHS]
\item[ESMF\_F90COMPILEPATHSLOCAL]
\item[ESMF\_F90COMPILER]
\item[ESMF\_F90COMPILERDEFAULT]
\item[ESMF\_F90ESMFLINKLIBS]
\item[ESMF\_F90IMOD]
\item[ESMF\_F90LINKER]
\item[ESMF\_F90LINKERDEFAULT]
\item[ESMF\_F90LINKLIBS]
\item[ESMF\_F90LINKOPTS]
\item[ESMF\_F90LINKPATHS]
\item[ESMF\_F90LINKRPATHS]
\item[ESMF\_F90MODDIR]
\item[ESMF\_F90OPTFLAG]
\item[ESMF\_F90OPTFLAG\_G]
\item[ESMF\_F90OPTFLAG\_O]
\item[ESMF\_F90OPTFLAG\_X]
\item[ESMF\_GREPV]
\item[ESMF\_INCDIR]
\item[ESMF\_INSTALL\_BINDIR]
\item[ESMF\_INSTALL\_BINDIR\_ABSPATH]
\item[ESMF\_INSTALL\_CMAKEDIR]
\item[ESMF\_INSTALL\_CMAKEDIR\_ABSPATH]
\item[ESMF\_INSTALL\_DOCDIR]
\item[ESMF\_INSTALL\_DOCDIR\_ABSPATH]
\item[ESMF\_INSTALL\_HEADERDIR]
\item[ESMF\_INSTALL\_HEADERDIR\_ABSPATH]
\item[ESMF\_INSTALL\_LIBDIR]
\item[ESMF\_INSTALL\_LIBDIR\_ABSPATH]
\item[ESMF\_INSTALL\_MODDIR]
\item[ESMF\_INSTALL\_MODDIR\_ABSPATH]
\item[ESMF\_INSTALL\_PREFIX]
\item[ESMF\_INSTALL\_PREFIX\_ABSPATH]
\item[ESMF\_LDIR]
\item[ESMF\_LIBDIR]
\item[ESMF\_LOCOBJDIR]
\item[ESMF\_MACHINE]
\item[ESMF\_MODDIR]
\item[ESMF\_MPIBATCHOPTIONS]
\item[ESMF\_MPILAUNCHOPTIONS]
\item[ESMF\_MPIMPMDRUN]
\item[ESMF\_MPIMPMDRUNDEFAULT]
\item[ESMF\_MPIRUN]
\item[ESMF\_MPIRUNDEFAULT]
\item[ESMF\_MPISCRIPTOPTIONS]
\item[ESMF\_MV]
\item[ESMF\_NO\_INTEGER\_1\_BYTE]
\item[ESMF\_NO\_INTEGER\_2\_BYTE]
\item[ESMF\_OS]
\item[ESMF\_OPTLEVEL]
\item[ESMF\_PTHREADS]
\item[ESMF\_PTHREADSDEFAULT]
\item[ESMF\_RANLIB]
\item[ESMF\_RANLIBDEFAULT]
\item[ESMF\_RM]
\item[ESMF\_RPATHPREFIX]
\item[ESMF\_SED]
\item[ESMF\_SEDDEFAULT]
\item[ESMF\_SITE]
\item[ESMF\_SITEDIR]
\item[ESMF\_SL\_LIBLIBS]
\item[ESMF\_SL\_LIBLINKER]
\item[ESMF\_SL\_LIBOPTS]
\item[ESMF\_SL\_LIBS\_TO\_MAKE]
\item[ESMF\_SL\_SUFFIX]
\item[ESMF\_STDIR]
\item[ESMF\_TEMPLATES]
\item[ESMF\_TESTDIR]
\item[ESMF\_TESTEXHAUSTIVE]
\item[ESMF\_TESTMPMD]
\item[ESMF\_TESTWITHTHREADS]
\item[ESMF\_UTCDIR]
\item[ESMF\_UTCSCRIPTS]
\item[ESMF\_WC]

\end{description}



\subsubsection{Customizing {\tt ESMC\_Conf.h} and {\tt ESMF\_Conf.inc}}

The {\tt ESMC\_Conf.h} file is used to define several settings used
during compilation of ESMF library code written in C++.

\begin{description}

\item[FTN\_X(func)]
Macro that will correctly expand "func" to match the Fortran symbol convention.
Use this macro for function names that contain an underscore.

\item[FTNX(func)]
Macro that will correctly expand "func" to match the Fortran symbol convention.
Use this macro for function names that do {\em not} contain an underscore.

\item[ESMCI\_FortranStrLenArg]
Typedef to match the data type of the 'hidden' string length argument that
Fortran uses when passing CHARACTER strings.

\item[ESMF\_PRESENT(arg)]
Macro for a boolean expression that returns TRUE if "arg" is a "present"
argument passed from Fortran into C++.

\item[ESMC\_POINTER\_SIZE]
Size of C pointer in bytes.

\end{description}


The {\tt ESMF\_Conf.inc} file is used to {\em optionally} define two 
important macros:

\begin{description}

\item[ESMF\_NO\_INITIALIZERS]
If this macro is defined ESMF will assume that initializers inside 
Fortran derived type definitions are not supported.

\item[ESMF\_SEQUENCE\_BUG]
If this macro is defined ESMF will not use the {\tt SEQUENCE} specifier
inside Fortran derived types under certain circumstances.

\end{description}


\subsection{Shared Object Libraries}

On many platforms, a shared object library is created in addition to the
standard {\tt .a} archive library.
Shared object libraries are libraries that are pre-linked into an executable.
They can then be linked to an application at run time.  There are many advantages
to using shared libraries.  These include smaller executable files, and shared
memory usage when multiple executables are running - as is often the case of
programs using MPI.  They also allow easier bug fixing and development because
the library can often be upgraded without necessarily re-linking the executables
which call into it.

Shared object libraries can be pre-linked to system libraries and using them
can simplify dealing with ESMF's dependency on Fortran90 and C++ runtime 
libraries.

See \ref{sec:ThirdParty} for third party library build requirements.

