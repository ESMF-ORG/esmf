% $Id$
`
\section{ESMF\_RegridWeightGen}
\label{sec:ESMF_RegridWeightGen}

\subsection{Description}

This section describes the offline regrid weight generation application provided by ESMF (for a description of ESMF regridding in general see Section~\ref{sec:regrid}). Regridding, also called remapping or interpolation, is the process of changing the grid that underlies data values while preserving qualities of the original data. Different kinds of transformations are appropriate for different problems. Regridding may be needed when communicating data between Earth system model components such as land and atmosphere, or between different data sets to support operations such as visualization. 

Regridding can be broken into two stages. The first stage is generation of an interpolation weight matrix that describes how points in
the source grid contribute to points in the destination grid. The second stage is the multiplication of values on the source grid by the
interpolation weight matrix to produce values on the destination grid. This is implemented as a parallel sparse matrix multiplication.

There are two options for accessing ESMF regridding functionality: integrated and offline. Integrated regridding is a process whereby interpolation
weights are generated via subroutine calls during the execution of the user's code. The integrated regridding can also perform the parallel sparse
matrix multiplication. In other words, ESMF integrated regridding allows a user to perform the whole process of interpolation within their code.
For a further description of ESMF integrated regridding please see Section~\ref{sec:fieldregrid}. 
In contrast to integrated regridding,
offline regridding is a process whereby interpolation weights are generated by a separate ESMF command line tool, not within the user code. The ESMF offline
regridding tool also only generates the interpolation matrix, the user is responsible for reading in this matrix and doing the actual interpolation
 (multiplication by the sparse matrix) in their code. The rest of this section further describes ESMF offline regridding.

For a discussion of installing and accessing ESMF command line tools such as this one please see the beginning of this part of the reference manual (Section~\ref{part:CLTs}) or for the quickest approach to just building and accessing the command line tools please refer to the "Building and using bundled ESMF Command Line Tools" Section in the ESMF User's Guide.

This application requires the NetCDF library to read the grid files and to write out the weight files in NetCDF format.  To compile ESMF with the NetCDF library, please refer to the "Third Party Libraries" Section in the ESMF User's Guide for more information.

As described above, this tool reads in
two grid files and outputs weights for interpolation
 between the two grids. The input and output files are all in NetCDF format. The grid files can be defined in five
different formats:  the SCRIP format~\ref{sec:fileformat:scrip} as is used as an input to SCRIP~\cite{ref:SCRIP}, the CF convension single-tile grid file~\ref{sec:fileformat:gridspec} following the
\htmladdnormallink{CF metadata conventions}{http://cfconventions.org}, the GRIDSPEC Mosaic file~\ref{sec:fileformat:mosaic} following the proposed \htmladdnormallink{GRIDSPEC standard}{http://extranet.gfdl.noaa.gov/~vb/gridstd/gridstd.html},  
the ESMF unstructured grid format~\ref{sec:fileformat:esmf} or the proposed CF unstructured grid data model (UGRID) ~\ref{sec:fileformat:ugrid}.  GRIDSPEC is a proposed CF extension for the annotation of complex Earth system grids.  In the latest ESMF library, we added support for multi-tile GRIDSPEC Mosaic file with non-overlapping tiles. For UGRID, we support the 2D flexible mesh topology with mixed triangles and quadrilaterals and fully 3D unstructured mesh topology with hexahedrons and tetrahedrons.  

In the latest ESMF implementation, the {\tt ESMF\_RegridWeightGen} command line toll can detect the type of the input grid files automatically.  The user
doesn't need to provide the source and destination grid file type arguments anymore.  The following arguments {\tt -t}, {\tt --src\_type}, {\tt --dst\_type}, {\tt --src\_meshname}, and {\tt --dst\_meshname} are no longer needed.  If provided, the application will simply ingore them. The rule to determine the file format is the following:

\begin{itemize}
  \item ESMF\_FILEFORMAT\_SCRIP:  variables grid\_corner\_lon and grid\_corner\_lat exists
  \item ESMF\_FILEFORMAT\_UGRID: a variable with attribute "cf\_role" or "standard\_name" set to "mesh\_topology"
  \item ESMF\_FILEFORMAT\_ESMFMESH: variables nodeCoords and elementConn exists
  \item ESMF\_FILEFORMAT\_CFGRID: variables with attributes "degree\_north" and "degree\_east" exists
  \item ESMF\_FILEFORMAT\_MOSAIC: a variable with attribute "standard\_name" set to "grid\_mosaic\_spec"
  \item ESMF\_FILEFORMAT\_TILE:  a varilable with attribute "standard\_name" set to "grid\_tile\_spec"
\end{itemize}


This command line tool can do regrid weight generation from a global or regional source grid to a global or regional destination grid.
As is true with many global models, this application currently assumes the latitude and longitude values refer to positions on a perfect sphere, as opposed to a more complex and accurate representation of the Earth's true shape such as would be used in a GIS system. (ESMF's current user base doesn't require this level of detail in representing the Earth's shape, but it could be added in the future if necessary.)

The interpolation weights generated by this application are output to a NetCDF file (specified by the "-w" or "\verb+--+weight"
keywords). Two type of weight files are supported: the SCRIP format is the same as that generated by SCRIP, see Section~\ref{sec:weightfileformat} for a description of the format; and a simple weight file containing only the weights and the source and destination grid indices (In ESMF
term, these are the {\tt factorList} and {\tt factorIndexList} generated by the ESMF weight calculation function {\tt ESMF\_FieldRegridStore()}.
Note that the sequence of the weights in the file can
vary with the number of processors used to run the application. This means that two weight files generated by using different
numbers of processors can contain exactly the same interpolation matrix, but can appear different in a direct line by line
comparison (such as would be done by ncdiff). The interpolation weights can be generated with
the bilinear, patch, nearest neighbor, first-order conservative, or second-order conservative methods described in Section~\ref{sec:rwg_regridmethods}.

        
Internally this application uses the ESMF public API to generate the interpolation weights.
If a source or destination grid is a single tile logically rectangular grid, then {\tt ESMF\_GridCreate()}
~\ref{sec:example:2DLogRecFromScrip}
is used to create an ESMF\_Grid object. The cell center
coordinates of the input grid are put into the center stagger location ({\tt ESMF\_STAGGERLOC\_CENTER}).
In addition, the corner coordinates are also put into the corner stagger location
({\tt ESMF\_STAGGERLOC\_CORNER}) for conservative regridding.  If a grid contains multiple logically rectangular tiles 
connected with each other by edges, such as a Cubed Sphere grid, the grid can be represented as a multi-tile ESMF\_Grid object created 
using {\tt ESMF\_GridCreateMosaic()}~\ref{sec:usage:cubedspherefromfile}. Such a grid is stored in the GRIDSPEC Mosaic and tile file format.~\ref{sec:fileformat:mosaic}
The method {\tt ESMF\_MeshCreate()} ~\ref{sec:example:UnstructFromFile}
is used to create an ESMF\_Mesh object, if the
source or destination grid is an unstructured grid. When making this call,
the flag {\tt convert3D} is set to {\tt TRUE} to convert the 2D coordinates into 3D Cartesian coordinates. 
Internally {\tt ESMF\_FieldRegridStore()} is used to generate the weight table and indices table representing the interpolation matrix.


\subsection{Regridding Options}\label{sec:rwg_options}

 The offline regrid weight generation application supports most of the options available in the rest of the ESMF regrid system. The following is a description of these options as relevant to the application. For a more in-depth description see Section~\ref{sec:regrid}.

\subsubsection{Poles}\label{sec:rwg_poles}
The regridding occurs in 3D to avoid
problems with periodicity and with the pole singularity. This application
supports four options for handling the pole region (i.e. the empty area above the top row of the source grid or below
the bottom row of the source grid).  Note that all of these pole options currently only work for logically rectangular grids (i.e. SCRIP format grids with grid\_rank=2 or GRIDSPEC single-tile format grids). The first option is to leave  the pole region empty ("-p none"), in this
case if a destination point lies above or below the
top row of the source grid, it will fail to map, yielding an error (unless "-i" is specified).
With the next two options, the pole region is handled by constructing
an artificial pole in the center of the top and bottom row of grid points and then filling
in the region from this pole to the edges of the source grid with triangles.
The pole is located at the average of the position of the points surrounding
it, but moved outward to be at the same radius as the rest of the points
in the grid. The difference between these two artificial pole options is what value is used at the pole.
 The default pole option ("-p all") sets the value at the pole to be the average of the values
of all of the grid points surrounding the pole. For the other option ("-p N"), the user chooses
a number N from 1 to the number of source grid points around the pole. For
each destination point, the value at the pole is then the average of the N source points
surrounding that destination point. For the last pole option ("-p teeth") no artificial pole is constructed, instead the
pole region is covered by connecting points across the top and bottom row of the source Grid into triangles. As
this makes the top and bottom of the source sphere flat, for a big enough difference between the size of
the source and destination pole regions, this can still result in unmapped destination points.
Only pole option "none" is currently supported with the conservative interpolation methods (e.g. "-m conserve") and with the
nearest neighbor interpolation methods ("-m nearestdtos" and "-m neareststod").

\subsubsection{Masking}\label{sec:rwg_masking}
       Masking is supported for both the logically rectangular grids and the unstructured grids.
If the grid file is in the SCRIP format, the variable "grid\_imask" is used as the mask.
If the value is set to 0 for a grid point, then that point is considered masked out and
won't be used in the weights generated by the application. If the grid file is in the ESMF format, the variable "element Mask" is used as the mask.  For a grid defined in the GRIDSPEC
single-tile or multi-tile grid or in the UGRID convention, there is no mask variable defined.
However, a GRIDSPEC single-tile file or a UGRID file may contain both the grid definition and the data.
The grid mask is usually constructed using the missing values defined in the data variable.
The regridding application provides the argument "\verb+--+src\_missingvalue" or
"\verb+--+dst\_missingvalue" for users to specify the variable name from where the mask can be
constructed.

\subsubsection{Extrapolation}\label{sec:rwg_extrap}
The {\tt ESMF\_RegridWeightGen} application supports a number of kinds of extrapolation to fill in points not mapped by the regrid method. 
Please see the sections starting with section~\ref{sec:extrapolation:overview} for a description of these methods. When using the application
an extrapolation method is specified by using the "\verb+--+extrap\_method" flag. For the inverse distance weighted average method (nearestidavg),
the number of source locations is specified using the "\verb+--+extrap\_num\_src\_pnts" flag, and the distance exponent is specified using 
the "\verb+--+extrap\_dist\_exponent" flag. For the creep fill method (creep), the number of creep levels is specified using the "\verb+--+extrap\_num\_levels" flag.

\subsubsection{Unmapped destination points}\label{sec:rwg_unmapped}
If a destination point can't be mapped, then the default behavior of the application is to stop with an error. By specifying "-i" or the equivalent "\verb+--+ignore\_unmapped " the user can cause the application to ignore unmapped destination points. In this case, the output matrix won't contain entries for the unmapped destination points. Note that the unmapped point detection doesn't
currently work for nearest destination to source method ("-m nearestdtos"), so when using that method it  is as if ``-i'' is always on.

\subsubsection{Line type}\label{sec:rwg_linetype}
 Another variation in the regridding supported with spherical grids is {\bf line type}. This is controlled by the "\verb+--+line\_type" or ``-l'' flag. This switch allows the user to select the path of the line which connects
two points on a sphere surface. This in turn controls the path along which distances are calculated and the shape of 
the edges that make up a cell. Both of these quantities can influence how interpolation weights are calculated, for example in
bilinear interpolation the distances are used to calculate the weights and the cell edges are used to determine to which source 
cell a destination point should be mapped. 

ESMF currently supports two line types: ``cartesian'' and ``greatcircle''. The ``cartesian'' option 
specifies that the line between two points follows a straight path through the 3D Cartesian space in which the sphere is embedded.
Distances are measured along  this 3D Cartesian line. Under this option cells are approximated by planes in 3D space, and their boundaries are 
3D Cartesian lines between their corner points.  The ``greatcircle'' option specifies that the line between two points follows
a great circle path along the sphere surface. (A great circle is the shortest path between two points on a sphere.) 
Distances are measured along the great circle path. Under this option cells are on the sphere surface, and their boundaries 
are great circle paths between their corner points. 


\subsection{Regridding Methods}\label{sec:rwg_regridmethods}
 This regridding application can be used to generate bilinear, patch, nearest neighbor, first-order conservative, or second-order conservative 
interpolation weights. The following is a description of these interpolation methods as relevant to the offline weight generation application. For a more in-depth description see Section~\ref{sec:regrid}.

\subsubsection{Bilinear}\label{sec:rwg_bilinear}
 The default interpolation method for the weight generation application is bilinear. The algorithm used by this application to 
generate the bilinear weights is the standard one found in many textbooks.  Each destination point is mapped to a location
in the source Mesh, the position of the destination point relative to the source points surrounding it is used to calculate the interpolation weights. A restriction on
bilinear interpolation is that ESMF doesn't support self-intersecting cells (e.g. a cell twisted into a bow tie) in the source grid.

\subsubsection{Patch}\label{sec:rwg_patch}
This application can also be used to generate patch interpolation weights. Patch
interpolation is the ESMF version of a technique called "patch recovery" commonly
used in finite element modeling~\cite{PatchInterp1}~\cite{PatchInterp2}. It typically results in better approximations to values and derivatives when compared to bilinear interpolation.
Patch interpolation works by constructing multiple polynomial patches to represent
the data in a source element. For 2D grids, these polynomials
are currently 2nd degree 2D polynomials. The interpolated value at the destination point
  is the weighted average of the values of the patches at that point.

The patch interpolation process works as follows.
For each source element containing a destination point
we construct a patch for each corner node that makes up the element (e.g. 4 patches for
quadrilateral elements, 3 for triangular elements). To construct a polynomial patch for
a corner node we gather all the elements around that node.
(Note that this means that the patch interpolation weights depends on the source
element's nodes, and the nodes of all elements neighboring the source element.)
We then use a least squares fitting algorithm to choose the set of coefficients
for the polynomial that produces the best fit for the data in the elements.
This polynomial will give a value at the destination point that fits the source data
in the elements surrounding the corner node. We then repeat this process for each
corner node of the source element generating a new polynomial for each set of elements.
To calculate the value at the destination point we do a weighted average of the values
of each of the corner polynomials evaluated at that point. The weight for a corner's
polynomial is the bilinear weight of the destination point with regard to that corner.

The patch method has a larger stencil than the bilinear, for this reason the patch weight matrix can be correspondingly larger
than the bilinear matrix (e.g. for a quadrilateral grid the patch matrix is around 4x the size of
 the bilinear matrix). This can be an issue when performing a regrid weight generation operation close to the memory
limit on a machine. 

The patch method does not guarantee that after regridding the range of values in the destination field is within the range of 
values in the source field. For example, if the mininum value in the source field is 0.0, then it's possible that after regridding with the 
patch method, the destination field will contain values less than 0.0.

This method currently doesn't support self-intersecting cells (e.g. a cell twisted into a bow tie) in the source grid.

\subsubsection{Nearest neighbor}\label{sec:rwg_nearest}
The nearest neighbor interpolation options work by associating a point in one set with the closest point in another set. If two points are equally
close then the point with the smallest index is arbitrarily used (i.e. the point with that would have the smallest index in the weight matrix). There are two
versions of this type of interpolation available in the regrid weight generation application. One of these is the nearest source to destination
method ("-m neareststod"). In this method each destination point is mapped to the closest source point. The other of these is the
nearest destination to source method ("-m nearestdtos"). In this method each source point is mapped to the closest destination point. Note
that with this method the unmapped destination point detection doesn't work, so no error will be returned even if there are destination points
which don't map to any source point.

\subsubsection{First-order conservative}\label{sec:rwg_conserve}
 The main purpose of this method is to preserve the integral of the field across the interpolation from source to destination.  
 (For a more in-depth description of what this preservation of the integral (i.e. conservation) means please see section~\ref{sec:rwg:conservation}.)  In this method the value across each source cell is treated as a constant, so it will typically have a larger 
 interpolation error than the bilinear or patch methods.  The first-order method used here is similar to that described in the following paper~\cite{ConservativeOrder1}.

 By default (or if "\verb+--+norm\_type dstarea"), the weight $w_{ij}$ for a particular source cell $i$ and destination cell $j$ are calculated as $w_{ij}=f_{ij} * A_{si}/A_{dj}$. 
In this equation $f_{ij}$ is the fraction of the source cell $i$ contributing to destination cell $j$, and $A_{si}$ and $A_{dj}$ are the areas of the source and 
destination cells. If "\verb+--+norm\_type fracarea", then the weights are further divided by the destination fraction. In other words, in that case $w_{ij}=f_{ij} * A_{si}/(A_{dj}*D_j)$ where $D_j$ is fraction of the destination cell that intersects the unmasked source grid. 

To see a description of how the different normalization options affect the values and integrals produced by the conservative methods see section~\ref{sec:rwg:conservative_norm_opts}. For a grid on a sphere this method uses great circle cells, for a description of potential problems with these see~\ref{sec:interpolation:great_circle_cells}.

\subsubsection{Second-order conservative}\label{sec:rwg_conserve2d}
 Like the first-order conservative method, this method's main purpose is to preserve the integral of the field across the interpolation from source to destination.  
 (For a more in-depth description of what this preservation of the integral (i.e. conservation) means please see section~\ref{sec:rwg:conservation}.)  The difference between the first and second-order conservative methods is that the second-order takes the source gradient into account, so
 it yields a smoother destination field that typically better matches the source field. This difference between the first and second-order methods 
 is particularly apparent when going from a coarse source grid to a finer destination grid. Another difference is that the second-order method
  does not guarantee that after regridding the range of values in the destination field is within the range of 
 values in the source field. For example, if the mininum value in the source field is 0.0, then it's possible that after regridding with the 
 second-order method, the destination field will contain values less than 0.0. The implementation of this method is based on the one described in this paper~\cite{ConservativeOrder2}.

  The weights for second-order are calculated in a similar manner to first-order~\ref{sec:rwg_conserve} with additional weights that take into account the gradient across the source cell. 

To see a description of how the different normalization options affect the values and integrals produced by the conservative methods see section~\ref{sec:rwg:conservative_norm_opts}. For a grid on a sphere this method uses great circle cells, for a description of potential problems with these see~\ref{sec:interpolation:great_circle_cells}.

\subsection{Conservation}\label{sec:rwg:conservation}
 Conservation means that the following equation will hold:
  $\sum^{all-source-cells}(V_{si}*A'_{si}) = \sum^{all-destination-cells}(V_{dj}*A'_{dj})$, where
 V is the variable being regridded and A is the area of a cell.  
 The subscripts s and d refer to source and destination values, and the i and j are the source
 and destination grid cell indices (flattening the arrays to 1 dimension). 

 There are a couple of options for how the areas (A) in the proceding equation can be calculated. By default, ESMF calculates the areas. For a grid on a sphere, 
areas are calculated by connecting the corner coordinates of each grid cell (obtained from the grid file) with great circles. For a Cartesian grid, areas are calculated
in the typcial manner for 2D polygons. If the user specifies the user area's option ("\verb+--+user\_areas"), then weights will be adjusted so that the equation above 
will hold for the areas provided in the grid files. In either case, the areas output to the weight file are the ones for which the weights have been adjusted to conserve.

\subsection{The effect of normalization options on integrals and values produced by conservative methods}\label{sec:rwg:conservative_norm_opts}
 It is important to note that by default (i.e. using destination area normalization) conservative regridding doesn't normalize the interpolation weights by the destination fraction. This means that for a destination grid which only partially overlaps the source grid the destination field which is output from the regrid operation should be divided by the corresponding destination fraction to yield the true interpolated values for cells which are only partially covered by the source grid.
The fraction also needs to be included when computing the total source and destination integrals. To include the fraction in the conservative weights, the user can specify the fraction area normalization type. This can be done by specifying "\verb+--+norm\_type fracarea'' on the command line. 

For weights generated using destination area normalization (either by not specifying any normalization type or by specifying "\verb+--+norm\_type dstarea"), 
the following pseudo-code shows how to adjust a destination field ({\tt dst\_field}) by the destination fraction ({\tt dst\_frac}) called {\tt frac\_b} in the weight file:

\begin{verbatim}
 for each destination element i
    if (dst_frac(i) not equal to 0.0) then
       dst_field(i)=dst_field(i)/dst_frac(i)
    end if
 end for
\end{verbatim}

For weights generated using destination area normalization (either by not specifying any normalization type or by specifying "\verb+--+norm\_type dstarea"), 
the following pseudo-code shows how to compute the total destination integral ({\tt dst\_total}) given the destination field values ({\tt dst\_field}) resulting
from the sparse matrix multiplication of the weights in the weight file by the source field, the destination area ({\tt dst\_area}) called {\tt area\_b} in the
weight file, and the destination fraction ({\tt dst\_frac}) called {\tt frac\_b} in the weight file. As in the previous paragraph, it also
shows how to adjust the destination field ({\tt dst\_field}) resulting from the sparse matrix multiplication by the fraction
({\tt dst\_frac}) called {\tt frac\_b} in the weight file:

\begin{verbatim}
 dst_total=0.0
 for each destination element i
    if (dst_frac(i) not equal to 0.0) then
       dst_total=dst_total+dst_field(i)*dst_area(i)
       dst_field(i)=dst_field(i)/dst_frac(i)
       ! If mass computed here after dst_field adjust, would need to be:
       ! dst_total=dst_total+dst_field(i)*dst_area(i)*dst_frac(i)
    end if
 end for
\end{verbatim}

For weights generated using fraction area normalization (set by specifying "\verb+--+norm\_type fracarea"), no adjustment of the destination field ({\tt dst\_field}) by the destination fraction is necessary. The following pseudo-code shows how to compute the total destination integral ({\tt dst\_total}) given the destination field values ({\tt dst\_field}) resulting
 from the sparse matrix multiplication of the weights in the weight file by the source field, the destination area ({\tt dst\_area}) called {\tt area\_b} in the
weight file, and the destination fraction ({\tt dst\_frac}) called {\tt frac\_b} in the weight file: 

\begin{verbatim}
 dst_total=0.0
 for each destination element i
       dst_total=dst_total+dst_field(i)*dst_area(i)*dst_frac(i)
 end for
\end{verbatim}

For either normalization type, the following pseudo-code shows how to compute the total source integral ({\tt src\_total}) given the source field values ({\tt src\_field}), the source area ({\tt src\_area}) called {\tt area\_a} in the weight file, and the source fraction ({\tt src\_frac}) called {\tt frac\_a} in the weight file:

\begin{verbatim}
 src_total=0.0
 for each source element i
    src_total=src_total+src_field(i)*src_area(i)*src_frac(i)
 end for
\end{verbatim}


\subsection{Usage}\label{sec:regridusage}

The command line arguments are all keyword based.  Both the long keyword prefixed with \verb+ '--' + or the
one character short keyword prefixed with {\tt '-'} are supported.  The format to run the application is
as follows:

\begin{verbatim}
ESMF_RegridWeightGen  
        --source|-s src_grid_filename
        --destination|-d dst_grid_filename
        --weight|-w out_weight_file
        [--method|-m bilinear|patch|nearestdtos|neareststod|conserve|conserve2nd]
        [--pole|-p none|all|teeth|1|2|..]
        [--line_type|-l cartesian|greatcircle]
        [--norm_type dstarea|fracarea]
        [--extrap_method none|neareststod|nearestidavg|nearestd|creep|creepnrstd]
        [--extrap_num_src_pnts <N>]
        [--extrap_dist_exponent <P>]
        [--extrap_num_levels <L>]
        [--ignore_unmapped|-i]
        [--ignore_degenerate]
        [-r]
        [--src_regional]
        [--dst_regional]
        [--64bit_offset]
        [--netcdf4]
        [--src_missingvalue var_name]
        [--dst_missingvalue var_name]
        [--src_coordinates lon_name,lat_name]
        [--dst_coordinates lon_name,var_name]
        [--tilefile_path filepath]
        [--src_loc center|corner]
        [--dst_loc center|corner]
        [--user_areas]
        [--weight_only]
        [--check]
        [--checkFlag]
        [--no_log]
        [--help]
        [--version]
        [-V]

where:
  --source or -s      - a required argument specifying the source grid
                        file name

  --destination or -d - a required argument specifying the destination
                        grid file name

  --weight or -w      - a required argument specifying the output regridding
                        weight file name

  --method or -m      - an optional argument specifying which interpolation
                        method is used. The value can be one of the following:

                        bilinear     - for bilinear interpolation, also the
                                       default method if not specified.
                        patch        - for patch recovery interpolation
                        neareststod  - for nearest source to destination interpolation
                        nearestdtos  - for nearest destination to source interpolation
                        conserve     - for first-order conservative interpolation
                        conserve2nd  - for second-order conservative interpolation

  --pole or -p        - an optional argument indicating how to extrapolate 
                        in the pole region. 
                        The value can be one of the following:

                        none  - No pole, the source grid ends at the top
                                (and bottom) row of nodes specified in
                                <source grid>.
                        all   - Construct an artificial pole placed in the
                                center of the top (or bottom) row of nodes,
                                but projected onto the sphere formed by the
                                rest of the grid. The value at this pole is
                                the average of all the pole values. This
                                is the default option.

                        teeth - No new pole point is constructed, instead
                                the holes at the poles are filled by
                                constructing triangles across the top and
                                bottom row of the source Grid. This can be
                                useful because no averaging occurs, however,
                                because the top and bottom of the sphere are
                                now flat, for a big enough mismatch between
                                 the size of the destination and source pole
                                regions, some destination points may still
                                not be able to be mapped to the source Grid.

                        <N>   - Construct an artificial pole placed in the
                                center of the top (or bottom) row of nodes,
                                but projected onto the sphere formed by the
                                rest of the grid. The value at this pole is
                                the average of the N source nodes next to
                                the pole and surrounding the destination
                                point (i.e.  the value may differ for each
                                destination point. Here N ranges from 1 to
                                the number of nodes around the pole.

    --line_type 
         or
         -l           - an optional argument indicating the type of path
                        lines (e.g. cell edges) follow on a spherical
                        surface. The default value depends on the regrid
                        method. For non-conservative methods the default is
                        cartesian. For conservative methods the default is
                        greatcircle. 

    --norm_type       - an optional argument indicating the type of normal-
                        ization to do when generating conservative weights. 
                        The default value is dstarea.

    --extrap_method   - an optional argument specifying which extrapolation
                        method is used to handle unmapped destination locations.
                        The value can be one of the following:

                        none         - no extrapolation method should be used.
                                       This is the default. 

                        neareststod  - nearest source to destination. Each
                                       unmapped destination location is mapped 
                                       to the closest source location. This 
                                       extrapolation method is not supported with 
                                       conservative regrid methods (e.g. conserve).
        
                        nearestidavg - inverse distance weighted average. 
                                       The value of each unmapped destination location
                                       is the weighted average of the closest N 
                                       source locations. The weight is the reciprocal 
                                       of the distance of the source from the destination
                                       raised to a power P. All the weights contributing 
                                       to one destination point are normalized so that 
                                       they sum to 1.0. The user can choose N and P by
                                       using --extrap_num_src_pnts and 
                                       --extrap_dist_exponent, but defaults are 
                                       also provided. This extrapolation method is not 
                                       supported with conservative regrid methods
                                       (e.g. conserve).

                        nearestd     - nearest mapped destination to 
                                       unmapped destination. Each
                                       unmapped destination location is mapped 
                                       to the closest mapped destination location. This 
                                       extrapolation method is not supported with 
                                       conservative regrid methods (e.g. conserve).

                        creep        - creep fill. 
                                       Here unmapped destination points are filled by 
                                       moving values from mapped locations to neighboring 
                                       unmapped locations. The value filled into a 
                                       new location is the average of its already filled
                                       neighbors' values. This process is repeated for 
                                       the number of levels indicated by the 
                                       --extrap_num_levels flag. This extrapolation
                                       method is not supported with conservative 
                                       regrid methods (e.g. conserve).

                        creepnrstd   - creep fill with nearest destination.  
                                       Here unmapped destination points are filled by 
                                       first doing a creep fill, and then filling the 
                                       remaining unmapped points by using 
                                       the nearest destination method (both of these
                                       methods are described in the entries above). 
                                       This extrapolation method is not supported 
                                       with conservative regrid methods (e.g. conserve).
                                       

    --extrap_num_src_pnts - an optional argument specifying how many source points
                            should be used when the extrapolation method is 
                            nearestidavg. If not specified, the default is 8.

    --extrap_dist_exponent - an optional argument specifying the exponent that 
                             the distance should be raised to when the 
                             extrapolation method is nearestidavg. If not specified, 
                             the default is 2.0.

    --extrap_num_levels - an optional argument specifying how many levels should
                          be filled for level based extrapolation methods (e.g. creep).

    --ignore_unmapped
           or
           -i         - ignore unmapped destination points. If not specified
                        the default is to stop with an error if an unmapped
                        point is found.

    --ignore_degenerate - ignore degenerate cells in the input grids. If not specified
                        the default is to stop with an error if an degenerate
                        cell is found.

    -r                - an optional argument specifying that the source and
                        destination grids are regional grids.  If the argument
                        is not given, the grids are assumed to be global.

    --src_regional    - an optional argument specifying that the source is
                        a regional grid and the destination is a global grid.

    --dst_regional    - an optional argument specifying that the destination
                        is a regional grid and the source is a global grid.

    --64bit_offset    - an optional argument specifying that the weight file
                        will be created in the NetCDF 64-bit offset format
                        to allow variables larger than 2GB.  Note the 64-bit
                        offset format is not supported in the NetCDF version
                        earlier than 3.6.0.  An error message will be generated
                        if this flag is specified while the application is
                        linked with a NetCDF library earlier than 3.6.0.

    --netcdf4         - an optional argument specifying that the output weight
                        will be created in the NetCDF4 format.  This option 
                        only works with NetCDF library version 4.1 and above 
                        that was compiled with the NetCDF4 file format enabled 
                        (with HDF5 compression). An error message will be 
                        generated if these conditions are not met.

    --src_missingvalue - an optional argument that defines the variable name 
                         in the source grid file if the file type is either CF Convension
                         single-tile or UGRID.  The regridder will generate a mask using 
                         the missing values of the data variable.  The missing 
                         value is defined using an attribute called "_FillValue" 
                         or "missing_value". 
     --dst_missingvalue - an optional argument that defines the variable name
                         in the destination grid file if the file type is
                         CF Convension single-tile or UGRID.  The regridder will generate a mask using
                         the missing values of the data variable.  The missing
                         value is defined using an attribute called "_FillValue"
                         or "missing_value"

    --src_coordinates - an optional argument that defines the longitude and
                        latitude variable names in the source grid file if
                        the file type is CF Convension single-tile.  The variable names are
                        separated by comma.  This argument is required in case
                        there are multiple sets of coordinate variables defined
                        in the file.  Without this argument, the offline regrid
                        application will terminate with an error message when
                        multiple coordinate variables are found in the file.

    --dst_coordinates - an optional argument that defines the longitude and
                        latitude variable names in the destination grid file
                        if the file type is CF Convension single-tile.  The variable names are
                        separated by comma.  This argument is required in case
                        there are multiple sets of coordinate variables defined
                        in the file.  Without this argument, the offline regrid
                        application will terminate with an error message when
                        multiple coordinate variables are found in the file.

    --tilefile_path   - the alternative file path for the tile files when either the source
                        or the destination grid is a GRIDSPEC Mosaic grid.  The path can
                        be either relative or absolute.  If it is relative, it is relative
                        to the working directory.  When specified, the gridlocation variable
                        defined in the Mosaic file will be ignored. 
                
    --src_loc         - an optional argument indicating which part of a source
                        grid cell to use for regridding. Currently, this flag is 
                        only required for non-conservative regridding when the 
                        source grid is an unstructured grid in ESMF or UGRID format.
                        For all other cases, only the center location is supported.
                        The value can be one of the following:

                        center - Regrid using the center location of each grid cell.

                        corner - Regrid using the corner location of each grid cell.

    --dst_loc         - an optional argument indicating which part of a destination
                        grid cell to use for regridding. Currently, this flag is 
                        only required for non-conservative regridding when the 
                        destination grid is an unstructured grid in ESMF or UGRID format.
                        For all other cases, only the center location is supported.
                        The value can be one of the following:

                        center - Regrid using the center location of each grid cell.

                        corner - Regrid using the corner location of each grid cell.


    --user_areas      - an optional argument specifying that the conservation
                        is adjusted to hold for the user areas provided in
                        the grid files. If not specified, then the 
                        conservation will hold for the ESMF calculated 
                        (great circle) areas.
                        Whichever areas the conservation holds for are output
                        to the weight file.

     --weight_only    - an optional argument specifying that the output weight file only 
                        contains the weights and the source and destination grid's indices.

     --check          - Check that the generated weights produce reasonable 
                        regridded fields. This is done by calling ESMF_Regrid() 
                        on an analytic source field using the weights generated 
                        by this application.  The mean relative error between 
                        the destination and analytic field is computed, as well 
                        as the relative error between the mass of the source and 
                        destination fields in the conservative case.

     --checkFlag      - Turn on more expensive extra error checking during 
                        weight generation.

     --no_log         - Turn off the ESMF Log files.  By default, ESMF creates 
                        multiple log files, one per PET.

     --help           - Print the usage message and exit.

     --version        - Print ESMF version and license information and exit.

     -V               - Print ESMF version number and exit.
\end{verbatim}


\subsection{Examples}

The example below shows the command to generate a set of conservative interpolation weights between a global
SCRIP format source grid file (src.nc) and a global SCRIP format destination grid file (dst.nc). The weights
are written into file w.nc. In this case the
ESMF library and applications have been compiled using an MPI parallel communication library
(e.g. setting ESMF\_COMM to openmpi) to enable it to run in parallel. To demonstrate running in parallel
the mpirun script is used to run the application in parallel on 4 processors.

\begin{verbatim}

  mpirun -np 4 ./ESMF_RegridWeightGen -s src.nc -d dst.nc -m conserve -w w.nc

\end{verbatim}

The next example below shows the command to do the same thing as the previous example except for three changes. The first
change is this time the source grid is regional ("{\tt \verb+--+src\_regional}"). The second change is that
for this example bilinear interpolation ("{\tt -m bilinear}") is being used. Because bilinear is the default, we could also
omit the "{\tt -m bilinear}". The third change is that in this example some of the destination points are expected to
not be found in the source grid, but the user is ok with that and just wants those points to not appear in the weight file instead of causing an error ("{\tt -i}").

\begin{verbatim}

  mpirun -np 4 ./ESMF_RegridWeightGen -i --src_regional -s src.nc -d dst.nc \
                 -m bilinear -w w.nc

\end{verbatim}

The last example shows how to use the missing values of a data variable to generate the
grid mask for a CF Convension single-tile file, how to specify the coordinate variable names
using "{\tt \verb+--+src\_coordinates}"
 and use user defined area for the conservative regridding.

\begin{verbatim}

  mpirun -np 4 ./ESMF_RegridWeightGen -s src.nc -d dst.nc -m conserve \
                 -w w.nc --src_missingvalue datavar \
                 --src_coordinates lon,lat --user_areas

\end{verbatim}

In the above example, "datavar" is the variable name defined in the source grid that will
 be used to construct the mask using its missing values.  In addition, "{\tt lon}" and "{\tt lat}" are the
variable names for the longitude and latitude values, respectively.


\subsection{Grid File Formats}

  This section describes the grid file formats supported by ESMF. These are typically used either to describe grids to ESMF\_RegridWeightGen or to create grids within ESMF. The following table summarizes the 
features supported by each of the grid file formats.

\begin{center}
\begin{tabular}{|l|c|c|c|c|c|}
\hline
Feature & SCRIP  & ESMF Unstruct. & CF TILE & UGRID & GRIDSPEC Mosaic\\
\hline
Create an unstructured Mesh            & YES & YES & NO  & YES & NO\\
Create a logically-rectangular Grid   & YES & NO  & YES & NO & YES\\
Create a multi-tile Grid      & NO  & NO  & NO  & NO & YES \\
2D                       & YES & YES & YES & YES & YES\\
3D                      & NO  & YES & NO  & YES & NO\\
Spherical coordinates         & YES & YES & YES & YES & YES\\
Cartesian coordinates         & NO  & YES & NO  & NO & NO\\
Non-conserv regrid on corners & NO  & YES & NO  & YES &YES\\
\hline
\end{tabular}
\label{fig:gridfileformatfeatures}
\end{center}


 The rest of this section contains a detailed descriptions of each grid file format along with a simple example of the format. 

\subsubsection{SCRIP Grid File Format}\label{sec:fileformat:scrip}

A SCRIP format grid file is a NetCDF file for describing grids. This format is the same as is used by the SCRIP~\cite{ref:SCRIP}
package, and so grid files which work with that package should also work here.  
When using the ESMF API, the file format flag {\tt ESMF\_FILEFORMAT\_SCRIP} can be used to indicate a file in this format.

SCRIP format files are capable of storing either 2D logically rectangular
grids or 2D unstructured grids. The basic format for both of these grids is the same and they are distinguished by the
value of the {\tt grid\_rank} variable. Logically rectangular grids have {\tt grid\_rank} set to 2,
whereas unstructured grids have this variable set to 1.

The following is a sample header of a logically rectangular grid file:

\begin{verbatim}
netcdf remap_grid_T42 {
dimensions:
      grid_size = 8192 ;
      grid_corners = 4 ;
      grid_rank = 2 ;

variables:
      int grid_dims(grid_rank) ;
      double grid_center_lat(grid_size) ;
         grid_center_lat:units = "radians";
      double grid_center_lon(grid_size) ;
         grid_center_lon:units = "radians" ;
      int grid_imask(grid_size) ;
         grid_imask:units = "unitless" ;
      double grid_corner_lat(grid_size, grid_corners) ;
         grid_corner_lat:units = "radians" ;
      double grid_corner_lon(grid_size, grid_corners) ;
         grid_corner_lon:units ="radians" ;

// global attributes:
         :title = "T42 Gaussian Grid" ;
}
\end{verbatim}

The {\tt grid\_size} dimension is the total number of cells in the grid; {\tt grid\_rank} refers to the
number of dimensions. In this case {\tt grid\_rank} is 2 for a 2D logically rectangular grid.
The integer array {\tt grid\_dims} gives the number of grid cells along each dimension.
The number of corners (vertices) in each grid cell is given by {\tt grid\_corners}.
The grid corner coordinates need to be listed in an order such that the corners are in counterclockwise
order.  Also, note that if your grid has a variable number of corners on grid cells, then
you should set {\tt grid\_corners} to be the highest value and use redundant points
on cells with fewer corners.

The integer array {\tt grid\_imask} is used to mask out grid cells which should
not participate in the regridding. The array values should be zero for any points
that do not participate in the regridding and one for all other points.
Coordinate arrays provide the latitudes and longitudes of cell centers
and cell corners. The unit of the coordinates can be either "{\tt radians}" or "{\tt degrees}".

Here is a sample header from a SCRIP unstructured grid file:

\begin{verbatim}
netcdf ne4np4-pentagons {
dimensions:
      grid_size = 866 ;
      grid_corners = 5 ;
      grid_rank = 1 ;
variables:
      int grid_dims(grid_rank) ;
      double grid_center_lat(grid_size) ;
         grid_center_lat:units = "degrees" ;
      double grid_center_lon(grid_size) ;
         grid_center_lon:units = "degrees" ;
      double grid_corner_lon(grid_size, grid_corners) ;
         grid_corner_lon:units = "degrees";
         grid_corner_lon:_FillValue = -9999. ;
      double grid_corner_lat(grid_size, grid_corners) ;
         grid_corner_lat:units = "degrees" ;
         grid_corner_lat:_FillValue = -9999. ;
      int grid_imask(grid_size) ;
         grid_imask:_FillValue = -9999. ;
      double grid_area(grid_size) ;
         grid_area:units = "radians^2" ;
         grid_area:long_name = "area weights" ;
}
\end{verbatim}

The variables are the same as described above, however, here {\tt grid\_rank = 1}. In this format there
is no notion of which cells are next to which, so to construct the unstructured mesh the connection between
cells is defined by searching for cells with the same corner coordinates. (e.g. the same {\tt grid\_corner\_lat}
and {\tt grid\_corner\_lon} values).

Both the SCRIP grid file format and the SCRIP weight file format work with the SCRIP 1.4 tools.

\subsubsection{ESMF Unstructured Grid File Format}\label{sec:fileformat:esmf}

ESMF supports a custom unstructured grid file format for describing meshes. This format is more compatible than the SCRIP format with the methods used to create an ESMF Mesh object, so less conversion needs to be done to create a Mesh. The ESMF format is thus more efficient than SCRIP when used with ESMF codes (e.g. the ESMF\_RegridWeightGen application). When using the ESMF API, the file format flag {\tt ESMF\_FILEFORMAT\_ESMFMESH} can be used to indicate a file in this format.

The following is a sample header in the ESMF format followed by a description:

\begin{verbatim}
netcdf mesh-esmf {
dimensions:
     nodeCount = 9 ;
     elementCount = 5 ;
     maxNodePElement = 4 ;
     coordDim = 2 ;
variables:
     double nodeCoords(nodeCount, coordDim);
            nodeCoords:units = "degrees" ;
     int elementConn(elementCount, maxNodePElement) ;
            elementConn:long_name = "Node Indices that define the element /
                                     connectivity";
            elementConn:_FillValue = -1 ;
            elementConn:start_index = 1 ;
     byte numElementConn(elementCount) ;
            numElementConn:long_name = "Number of nodes per element" ;
     double centerCoords(elementCount, coordDim) ;
            centerCoords:units = "degrees" ;
     double elementArea(elementCount) ;
            elementArea:units = "radians^2" ;
            elementArea:long_name = "area weights" ;
     int elementMask(elementCount) ;
            elementMask:_FillValue = -9999. ;
// global attributes:
            :gridType="unstructured";
            :version = "0.9" ;
\end{verbatim}

 In the ESMF format the NetCDF dimensions have the following meanings. The {\tt nodeCount} dimension is the number of nodes in the mesh.
 The {\tt elementCount} dimension is the number of elements in the mesh. The {\tt maxNodePElement} dimension is the maximum number
 of nodes in any element in the mesh. For example, in a mesh containing just triangles, then {\tt maxNodePElement} would be 3. However,
 if the mesh contained one quadrilateral then {\tt maxNodePElement} would need to be 4. The {\tt coordDim} dimension is the number of dimensions
 of the points making up the mesh (i.e. the spatial dimension of the mesh). For example, a 2D planar mesh would have {\tt coordDim} equal to 2.  

 In the ESMF format the NetCDF variables have the following meanings. The {\tt nodeCoords} variable contains the coordinates for each node.
 {\tt nodeCoords} is a two-dimensional array of dimension {\tt (nodeCount,coordDim)}.
 For a 2D Grid, {\tt coordDim} is 2 and the grid can be either spherical or Cartesian. If the {\tt units}
 attribute is either {\tt degrees} or {\tt radians}, it is spherical. {\tt nodeCoords(:,1)} contains 
the longitude coordinates and {\tt nodeCoords(:,2)} contains the latitude coordinates.  If the value of 
the {\tt units} attribute is {\tt km}, {\tt kilometers} or {\tt meters}, the grid is in 2D Cartesian 
coordinates. {\tt nodeCoords(:,1)} contains the x coordinates and
 {\tt nodeCoords(:,2)} contains the y coordinates.
 The same order applies to {\tt centerCoords}.
 For a 3D Grid, {\tt coordDim} is 3 and the grid is assumed to be Cartesian. {\tt nodeCoords(:,1)} contains the x coordinates, {\tt nodeCoords(:,2)} contains the y coordinates, 
 and {\tt nodeCoords(:,3)} contains the z coordinates.  The same order applies to {\tt centerCoords}.
A 2D grid in the Cartesian coordinate can only be regridded into another 2D grid in the Cartesian coordinate.
 
 The {\tt elementConn} variable describes how the nodes are connected together to form each element. For each element, this variable contains a list of indices into the {\tt nodeCoords} variable pointing to the nodes which make up that
 element. By default, the index is 1-based.  It can be changed to 0-based by adding an attribute 
{\tt start\_index} of value 0 to the {\tt elementConn} variable.  The order of the indices describing the element is important.
 The proper order for elements available in an ESMF mesh can be found in Section~\ref{const:meshelemtype}. The file format does support 2D polygons with more
 corners than those in that section, but internally these are broken into triangles. For these polygons, the corners should
 be listed such that they are in counterclockwise order around the element.
{\tt elementConn} can be either a 2D array or a 1D array.  If it is a 2D array, the second
 dimension of the {\tt elementConn} variable has to be the size of the largest number of nodes in any element (i.e. {\tt maxNodePElement}), the actual number of
 nodes in an element is given by the {\tt numElementConn} variable. For a given dimension (i.e. {\tt coordDim}) the number of nodes in the element
 indicates the element shape. For example in 2D, if {\tt numElementConn} is 4 then the element is a quadrilateral. In 3D, if {\tt numElementConn} is 8
 then the element is a hexahedron.

If the grid contains some elements with large number of edges, using a 2D array for {\tt elementConn} could take a lot of space.  
In that case, {\tt elementConn} can be represented as a 1D array that stores the edges of all the elements continuously.  When {\tt elementConn} is a 1D array, the dimension {\tt maxNodePElement} is no longer needed, instead, a new dimension variable {\tt connectionCount}
is required to define the size of {\tt elementConn}.  The value of {\tt connectionCount} is the sum of all the values in {\tt numElementConn}.

The following is an example grid file using 1D array for {\tt elementConn}:

\begin{verbatim}
netcdf catchments_esmf1 {
dimensions:
        nodeCount = 1824345 ;
        elementCount = 68127 ;
        connectionCount = 18567179 ;
        coordDim = 2 ;
variables:
        double nodeCoords(nodeCount, coordDim) ;
                nodeCoords:units = ``degrees'' ;
        double centerCoords(elementCount, coordDim) ;
                centerCoords:units = ``degrees'' ;
        int elementConn(connectionCount) ;
                elementConn:polygon_break_value = -8 ;
                elementConn:start_index = 0. ;
        int numElementConn(elementCount) ;
}
\end{verbatim}

In some cases, one mesh element may contain multiple polygons and these polygons are separated by a special value defined in the attribute
 {\tt polygon\_break\_value}. 
   
 The rest of the variables in the format are optional. The {\tt centerCoords} variable gives the coordinates of the center of the corresponding element.
 This variable is used by ESMF for non-conservative interpolation on the data field residing at the center of the elements.  The {\tt elementArea} variable gives the area (or volume in 3D) of the corresponding element. This
 area is used by ESMF during conservative interpolation. If not specified, ESMF calculates the area (or volume) based on the coordinates of the nodes
 making up the element. The final variable is the {\tt elementMask} variable. This variable allows the user to specify a mask value for
 the corresponding element. If the value is 1, then the element is unmasked and if the value is 0 the element is masked.
 If not specified, ESMF assumes that no elements are masked.

The following is a picture of a small example mesh and a sample ESMF format header using non-optional variables describing that mesh:

\begin{verbatim}
  2.0   7 ------- 8 ------- 9
        |         |         |
        |    4    |    5    |
        |         |         |
  1.0   4 ------- 5 ------- 6
        |         |  \   3  |
        |    1    |    \    |
        |         |  2   \  |
  0.0   1 ------- 2 ------- 3

       0.0       1.0        2.0

        Node indices at corners
       Element indices in centers

netcdf mesh-esmf {
dimensions:
        nodeCount = 9 ;
        elementCount = 5 ;
        maxNodePElement = 4 ;
        coordDim = 2 ;
variables:
        double  nodeCoords(nodeCount, coordDim);
                nodeCoords:units = "degrees" ;
        int elementConn(elementCount, maxNodePElement) ;
                elementConn:long_name = "Node Indices that define the element /
                                         connectivity";
                elementConn:_FillValue = -1 ;
        byte numElementConn(elementCount) ;
                numElementConn:long_name = "Number of nodes per element" ;
// global attributes:
                :gridType="unstructured";
                :version = "0.9" ;
data:
    nodeCoords=
        0.0, 0.0,
        1.0, 0.0,
        2.0, 0.0,
        0.0, 1.0,
        1.0, 1.0,
        2.0, 1.0,
        0.0, 2.0,
        1.0, 2.0,
        2.0, 2.0 ;

    elementConn=
        1, 2, 5,  4,
        2, 3, 5, -1,
        3, 6, 5, -1,
        4, 5, 8,  7,
        5, 6, 9,  8 ;

    numElementConn= 4, 3, 3, 4, 4 ;
}

\end{verbatim}

\subsubsection{CF Convention Single Tile File Format}\label{sec:fileformat:gridspec}

ESMF\_RegridWeightGen supports single tile logically rectangular lat/lon grid files that follow the NETCDF CF convention based on
\htmladdnormallink{CF Metadata Conventions V1.6} {http://cfconventions.org/cf-conventions/v1.6.0/cf-conventions.html}. When using the ESMF API, the file format flag {\tt ESMF\_FILEFORMAT\_CFGRID} (or its equivalent {\tt ESMF\_FILEFORMAT\_GRIDSPEC}) can be used to indicate a file in this format.  

 An example grid file is shown below.
The cell center coordinate variables are determined by the value of its attribute {\tt units}.  The longitude
variable has the attribute value set to either {\tt degrees\_east}, {\tt degree\_east}, {\tt degrees\_E}, {\tt degree\_E},
{\tt degreesE} or {\tt degreeE}.  The latitude variable has the attribute value set to {\tt degrees\_north}, {\tt degree\_north}, {\tt degrees\_N},
{\tt degree\_N}, {\tt degreesN} or {\tt degreeN}.   The latitude and the longitude variables are one-dimensional arrays if the grid is a regular lat/lon grid, two-dimensional arrays if the grid is curvilinear. The bound coordinate
variables define the bound or the corner coordinates of a cell.  The bound variable name is specified in the
{\tt bounds} attribute of the latitude and longitude variables.  In the following example, the latitude bound
variable is {\tt lat\_bnds} and the longitude bound variable is {\tt lon\_bnds}.  The bound variables are 2D
arrays for a regular lat/lon grid and a 3D array for a curvilinear grid.  The first dimension of the bound
array is 2 for a regular lat/lon grid and 4 for a curvilinear grid.  The bound coordinates for a curvilinear
grid are defined in counterclockwise order. Since the grid is a regular lat/lon grid,
the coordinate variables are 1D and the bound variables are 2D with the first dimension equal to 2.
The bound coordinates will be read in and stored in a ESMF Grid object as the corner stagger coordinates when doing a conservative regrid.  In case there are multiple sets of coordinate variables defined in a grid file,
the offline regrid application will return an error for duplicate latitude or longitude variables unless
"{\tt \verb+--+src\_coordinates}" 
or "{\tt \verb+--+src\_coordinates}" options are used to specify the coordinate variable names
to be used in the regrid.

\begin{verbatim}
netcdf single_tile_grid {
dimensions:
	time = 1 ;
	bound = 2 ;
	lat = 181 ;
	lon = 360 ;
variables:
	double lat(lat) ;
		lat:bounds = "lat_bnds" ;
		lat:units = "degrees_north" ;
		lat:long_name = "latitude" ;
		lat:standard_name = "latitude" ;
	double lat_bnds(lat, bound) ;
	double lon(lon) ;
		lon:bounds = "lon_bnds" ;
		lon:long_name = "longitude" ;
		lon:standard_name = "longitude" ;
		lon:units = "degrees_east" ;
	double lon_bnds(lon, bound) ;
	float so(time, lat, lon) ;
		so:standard_name = "sea_water_salinity" ;
		so:units = "psu" ;
		so:missing_value = 1.e+20f ;
}
\end{verbatim}

2D Cartesian coordinates can be supplied in additional to the required
longitude/latitude coordinates.  They can be used in ESMF to create a grid and
used in ESMF\_RegridWeightGen.   The Cartesian coordinate variables have to
include an "{\tt axis}" attribute with value "X" or "Y".  The "{\tt units}"
attribute can be either "m" or  "meters" for meters or  "km" or  "kilometers"
for kilometers.  When a grid with 2D Cartesian coordinates are used in
ESMF\_RegridWeightGen, the optional arguments "{\tt \verb+--+src\_coordinates}" 
or "{\tt \verb+--+src\_coordinates}" have to be used to specify the coordinate
variable names.  A grid with 2D Cartesian coordinates can only be regridded
with another grid in 2D Cartesian coordinates.  Internally in ESMF, the
Cartesian coordinates are all converted into kilometers.  Here is an example
of the 2D Cartesian coordinates:

\begin{verbatim}
      double xc(xc) ;
              xc:long_name = "x-coordinate in Cartesian system" ;
              xc:standard_name = "projection_x_coordinate" ;
              xc:axis = "X" ;
              xc:units = "m" ;
      double yc(yc) ;
              yc:long_name = "y-coordinate in Cartesian system" ;
              yc:standard_name = "projection_y_coordinate" ;
              yc:axis = "Y" ;
              yc:units = "m" ;
\end{verbatim}


Since a CF convension tile file does not have a way to specify the grid mask, the mask is usually derived by the missing values stored in a data variable.  ESMF\_RegridWeightGen provides an option for users to
derive the grid mask from a data variable's missing values.  The value of the missing value is defined by the
variable attribute {\tt missing\_value} or {\tt \_FillValue}.  If the value of the data point is equal to the
missing value, the grid mask for that grid point is set to 0, otherwise, it is set to 1.   In the following
grid, the variable {\tt so} can be used to derive the grid mask.  A data variable could be a 2D, 3D or 4D.
For example, it may have additional depth and time dimensions.
It is assumed that the first and the second dimensions of the data variable should be the longitude and the
latitude dimension.  ESMF\_RegridWeightGen will use the first 2D data values to derive the grid mask.


\subsubsection{CF Convention UGRID File Format}\label{sec:fileformat:ugrid}

ESMF\_RegridWeightGen supports NetCDF files that follow the UGRID conventions for unstructured grids.

The UGRID file format is a proposed extension to the CF metadata conventions for the unstructured grid data model. The latest proposal can be found at \htmladdnormallink{https://github.com/ugrid-conventions/ugrid-conventions}{https://github.com/ugrid-conventions/ugrid-conventions}.  The proposal is still evolving, the Mesh creation API and ESMF\_RegridWeightGen in the current ESMF release is based on UGRID Version 0.9.0 published on October 29, 2013. When using the ESMF API, the file format flag {\tt ESMF\_FILEFORMAT\_UGRID} can be used to indicate a file in this format.

In the UGRID proposal, a 1D, 2D, or 3D mesh topology can be defined for an unstructured grid.  Currently, ESMF
supports two types of meshes: (1) the 2D flexible mesh topology where each cell (a.k.a. "face" as defined in the UGRID document) in the mesh is either a triangle or a quadrilateral, and (2) the fully 3D unstructured mesh topology where each cell (a.k.a. "volume" as defined in the UGRID document) in the mesh
is either a tetrahedron or a hexahedron.  Pyramids and wedges are not currently supported in ESMF, but they
can be defined as degenerate hexahedrons.   ESMF\_RegridWeightGen also
supports UGRID 1D network mesh topology in a limited way:  A 1D mesh in UGRID
can be used as the source grid for nearest neighbor regridding, and as the
destination grid for non-conservative regridding.  

The main addition of the UGRID extension is a dummy variable that defines the mesh
topology.  This additional variable has a required attribute {\tt cf\_role}
with value {\tt "mesh\_topology"}.  In addition, it has two more required attributes: {\tt topology\_dimension}
and {\tt node\_coordinates}.  If it is a 1D mesh, {\tt topology\_dimension} is
set to 1.  
If it is a 2D mesh (i.e., {\tt topology\_dimension} equals to 2), an additional attribute
{\tt face\_node\_connectivity} is required.  If it is a 3D mesh (i.e., {\tt topology\_dimension} equals to 3), two additional attributes {\tt volume\_node\_connectivity} and {\tt volume\_shape\_type} are required.
The value of attribute {\tt node\_coordinates} is a list of the names of the node longitude and latitude variables,
plus the elevation variable if it is a 3D mesh.
The value of attribute {\tt face\_node\_connectivity} or {\tt volume\_node\_connectivity} is the variable name that defines the corner node indices for each mesh cell. The additional attribute {\tt volume\_shape\_type} for the
3D mesh points to a flag variable that specifies the shape type of each cell in the mesh.

Below is a sample 2D mesh called {\tt FVCOM\_grid2d}. The dummy mesh topology variable is {\tt fvcom\_mesh}.  As described above, its {\tt cf\_role} attribute has to be {\tt mesh\_topology}
and the {\tt topology\_dimension} attribute has to be 2 for a 2D mesh.  It defines
the node coordinate variable names to be {\tt lon} and {\tt lat}.  It also specifies the face/node connectivity variable name as {\tt nv}.

The variable {\tt nv} is a two-dimensional array that defines the node indices of each face. The first dimension
defines the maximal number of nodes for each face. In this example, it is a
triangle mesh so the number of nodes per face is 3.  Since each face may have a different number of corner nodes,
some of the cells may have fewer nodes than the specified dimension. In that case, it is filled with the
missing values defined by the attribute {\tt \_FillValue}.  If {\tt \_FillValue} is not defined, the default value
is -1. The nodes are in counterclockwise order.  An optional attribute
{\tt start\_index} defines whether the node index is 1-based or 0-based.  If {\tt start\_index} is not defined, 
the default node index is 0-based.

The coordinate variables follows the CF metadata convention for coordinates.  They are 1D array with attribute
{\tt standard\_name} being either {\tt latitude} or {\tt longitude}.  The units of the coordinates can be either {\tt degrees} or {\tt radians}.

The UGRID files may also contain data variables.  The data may be located at the nodes or at the faces.  Two additional attributes are introduced in the UGRID extension for the data variables:  {\tt location} and {\tt mesh}.  The {\tt location}
attribute defines where the data is located, it can be either {\tt face} or {\tt node}.  The {\tt mesh} attribute defines which mesh topology this variable belongs to since multiple mesh topologies may be defined in one
file.  The {\tt coordinates} attribute defined in the CF conventions can also be used to associate the variables to their locations.  ESMF checks both {\tt location} and {\tt coordinates} attributes to determine where the data variable is defined upon. If both attributes are present, the {\tt location} attribute takes the precedence.  ESMF\_RegridWeightGen uses the data variable on the face to derive the element masks for the mesh cell and variable on the node to derive the node masks for the mesh.

When creating a ESMF Mesh from a UGRID file, the user has to provide the mesh topology variable name to {\tt ESMF\_MeshCreate()}.

\begin{verbatim}
netcdf FVCOM_grid2d {
dimensions:
	node = 417642 ;
	nele = 826866 ;
	three = 3 ;
        time  = 1 ;

variables:
// Mesh topology
	int fvcom_mesh;
		fvcom_mesh:cf_role = "mesh_topology" ;
		fvcom_mesh:topology_dimension = 2. ;
		fvcom_mesh:node_coordinates = "lon lat" ;
		fvcom_mesh:face_node_connectivity = "nv" ;
	int nv(nele, three) ;
		nv:standard_name = "face_node_connectivity" ;
		nv:start_index = 1. ;

// Mesh node coordinates
	float lon(node) ;
                lon:standard_name = "longitude" ;
	        lon:units = "degrees_east" ;
	float lat(node) ;
                lat:standard_name = "latitude" ;
		lat:units = "degrees_north" ;

// Data variable
	float ua(time, nele) ;
		ua:standard_name = "barotropic_eastward_sea_water_velocity" ;
		ua:missing_value = -999. ;
		ua:location = "face" ;
		ua:mesh = "fvcom_mesh" ;
	float va(time, nele) ;
		va:standard_name = "barotropic_northward_sea_water_velocity" ;
		va:missing_value = -999. ;
		va:location = "face" ;
		va:mesh = "fvcom_mesh" ;
}
\end{verbatim}

Following is a sample 3D UGRID file containing hexahedron cells. The dummy mesh topology variable is {\tt fvcom\_mesh}. Its {\tt cf\_role} attribute has to be {\tt mesh\_topology}
and {\tt topology\_dimension} attribute has to be 3 for a 3D mesh.  There are two additional required attributes:
{\tt volume\_node\_connectivity} specifies a variable name that defines the corner indices of the mesh cells and
{\tt volume\_shape\_type} specifies a variable name that defines the type of the mesh cells.

The node coordinates are defined by variables {\tt nodelon}, {\tt nodelat} and {\tt height}. Currently, the units
attribute for the height variable is either {\tt kilometers}, {\tt km} or {\tt meters}.
The variable {\tt vertids} is a two-dimensional array that defines the corner node indices of each mesh cell. The first dimension
defines the maximal number of nodes for each cell. There is only one type of cells in the sample grid, i.e. hexahedrons, so the maximal number
of nodes is 8.  The node order is defined in ~\ref{const:meshelemtype}.  The index can be either 1-based or 0-based and
the default is 0-based.
 Setting an optional attribute {\tt start\_index} to 1 changed it to 1-based index scheme.
The variable {\tt meshtype} is a one-dimensional integer array that defines the shape type of each cell.  Currently, ESMF only
supports tetrahedron and hexahedron shapes. There are three attributes in {\tt meshtype}: {\tt flag\_range}, {\tt flag\_values}, and {\tt flag\_meanings} representing the range of the flag values, all the possible flag values, and the meaning of each flag value, respectively.  {\tt flag\_range} and {\tt flag\_values} are either a scalar or an array of integers.  {\tt flag\_meanings} is a text string containing a list of shape types separated by space. In this example, there
is only one shape type, thus, the values of {\tt meshtype} are all 1.

\begin{verbatim}
netcdf wam_ugrid100_110 {
dimensions:
	nnodes = 78432 ;
	ncells = 66030 ;
	eight = 8 ;
variables:
	int mesh ;
		mesh:cf_role = "mesh_topology" ;
		mesh:topology_dimension = 3. ;
		mesh:node_coordinates = "nodelon nodelat height" ;
		mesh:volume_node_connectivity = "vertids" ;
		mesh:volume_shape_type = "meshtype" ;
	double nodelon(nnodes) ;
		nodelon:standard_name = "longitude" ;
		nodelon:units = "degrees_east" ;
	double nodelat(nnodes) ;
		nodelat:standard_name = "latitude" ;
		nodelat:units = "degrees_north" ;
	double height(nnodes) ;
		height:standard_name = "elevation" ;
		height:units = "kilometers" ;
	int vertids(ncells, eight) ;
		vertids:cf_role = "volume_node_connectivity" ;
		vertids:start_index = 1. ;
	int meshtype(ncells) ;
		meshtype:cf_role = "volume_shape_type" ;
		meshtype:flag_range = 1. ;
		meshtype:flag_values = 1. ;
		meshtype:flag_meanings = "hexahedron" ;
}
\end{verbatim}

\subsubsection{GRIDSPEC Mosaic File Format}\label{sec:fileformat:mosaic}

GRIDSPEC is a draft proposal to extend the Climate and Forecast (CF) metadata conventions for the representation of gridded data for Earth System Models.  The original GRIDSPEC standard was proposed by V. Balaji and Z. Liang of GFDL (see \htmladdnormallink{ref} {http://www.gfdl.noaa.gov/~vb/gridstd/gridstd.html}). GRIDSPEC extends the current CF convention to support grid  mosaics, i.e., a grid consisting of multiple logically
rectangular grid tiles. It also provides a mechanism for storing a grid dataset in multiple files.  Therefore,
it introduces different types of files, such as a mosaic file that defines the multiple tiles and their
connectivity, and a tile file for a single tile grid definition on a so-called "Supergrid" format. When using the ESMF API, the file format flag {\tt ESMF\_FILEFORMAT\_MOSAIC} can be used to indicate a file in this format.

Following is an example of a mosaic file that defines a 6 tile Cubed Sphere grid:

\begin{verbatim}
netcdf C48_mosaic {
dimensions:
	ntiles = 6 ;
	ncontact = 12 ;
	string = 255 ;
variables:
	char mosaic(string) ;
		mosaic:standard_name = "grid_mosaic_spec" ;
		mosaic:children = "gridtiles" ;
		mosaic:contact_regions = "contacts" ;
		mosaic:grid_descriptor = "" ;
	char gridlocation(string) ;
	char gridfiles(ntiles, string) ;
	char gridtiles(ntiles, string) ;
	char contacts(ncontact, string) ;
		contacts:standard_name = "grid_contact_spec" ;
		contacts:contact_type = "boundary" ;
		contacts:alignment = "true" ;
		contacts:contact_index = "contact_index" ;
		contacts:orientation = "orient" ;
	char contact_index(ncontact, string) ;
		contact_index:standard_name = "starting_ending_point_index_of_contact" ;

data:

mosaic = "C48_mosaic" ;

gridlocation = "./data/" ;

gridfiles =
  "horizontal_grid.tile1.nc",
  "horizontal_grid.tile2.nc",
  "horizontal_grid.tile3.nc",
  "horizontal_grid.tile4.nc",
  "horizontal_grid.tile5.nc",
  "horizontal_grid.tile6.nc" ;

gridtiles =
  "tile1",
  "tile2",
  "tile3",
  "tile4",
  "tile5",
  "tile6" ;

contacts =
  "C48_mosaic:tile1::C48_mosaic:tile2",
  "C48_mosaic:tile1::C48_mosaic:tile3",
  "C48_mosaic:tile1::C48_mosaic:tile5",
  "C48_mosaic:tile1::C48_mosaic:tile6",
  "C48_mosaic:tile2::C48_mosaic:tile3",
  "C48_mosaic:tile2::C48_mosaic:tile4",
  "C48_mosaic:tile2::C48_mosaic:tile6",
  "C48_mosaic:tile3::C48_mosaic:tile4",
  "C48_mosaic:tile3::C48_mosaic:tile5",
  "C48_mosaic:tile4::C48_mosaic:tile5",
  "C48_mosaic:tile4::C48_mosaic:tile6",
  "C48_mosaic:tile5::C48_mosaic:tile6" ;

 contact_index =
  "96:96,1:96::1:1,1:96",
  "1:96,96:96::1:1,96:1",
  "1:1,1:96::96:1,96:96",
  "1:96,1:1::1:96,96:96",
  "1:96,96:96::1:96,1:1",
  "96:96,1:96::96:1,1:1",
  "1:96,1:1::96:96,96:1",
  "96:96,1:96::1:1,1:96",
  "1:96,96:96::1:1,96:1",
  "1:96,96:96::1:96,1:1",
  "96:96,1:96::96:1,1:1",
  "96:96,1:96::1:1,1:96" ;
}
\end{verbatim}
 
A GRIDSPEC Mosaic file is identified by a dummy variable with its {\tt standard\_name} attribute set to {\tt grid\_mosaic\_spec}.
The {\tt children} attribute of this dummy variable provides the variable name that contains the tile names and the 
{\tt contact\_region} attribute points to the variable name that defines a list of tile pairs that are connected
to each other.  For a Cubed Sphere grid, there are six tiles and 12 connections.  The {\tt contacts} variable, the 
variable that defines the contact\_region
has three required attributes: {\tt standard\_name}, {\tt contact\_type}, and {\tt contact\_index}.  {\tt startand\_name}
has to be set to {\tt grid\_contact\_spec}.  {\tt contact\_type} can be either {\tt boundary} or {\tt overlap}. Currently, ESMF
only supports non-overlapping tiles connected by {\tt boundary}. {\tt contact\_index} defines the variable name that contains the information defining how the
two adjacent tiles are connected to each other.  In the above example, the {\tt contact\_index} variable contains 12 entries.  Each entry
contains the index of four points that defines the two edges that contact to 
 each other from the two neighboring tiles.  Assuming the four points are A, B, C, and D.  
 A and B defines the edge of tile 1 and C and D defines the edge of tile 2.  A is the same point
 as C and B is the same as D.  (Ai, Aj) is the index for point A. The entry looks like this:
\begin{verbatim}
  Ai:Bi,Aj:Bj::Ci:Di,Cj:Dj
\end{verbatim}

There are two fixed-name variables required in the mosaic file: variable {\tt gridfiles} defines the associated tile file names and 
variable {\tt gridlocation} defines the directory path of the tile files.
The {\tt gridlocation} can be overwritten with an command line argument {\tt --tilefile\_path} in ESMF\_RegridWeightGen application.  

It is possible to define a single-tile Mosaic file.  If there is only one tile in the Mosaic, the {\tt contact\_region} attribute in the
{\tt grid\_mosaic\_spec} varilable will be ignored.

Each tile in the Mosaic is a logically rectangular lat/lon grid and is defined in a separate file.   The tile file used in the GRIDSPEC Mosaic file defines the coordinates of a so-called 
{\tt supergrid}.  A supergrid contains all the
stagger locations in one grid.  It contains the corner, edge and center coordinates all in one 2D array.
In this example, there are 48 elements in each side of a tile, therefore, the size of the supergrid is 
48*2+1=97, i.e. 97x97.

Here is the header of one of the tile files:

\begin{verbatim}
netcdf horizontal_grid.tile1 {
dimensions:
	string = 255 ;
	nx = 96 ;
	ny = 96 ;
	nxp = 97 ;
	nyp = 97 ;
variables:
	char tile(string) ;
		tile:standard_name = "grid_tile_spec" ;
		tile:geometry = "spherical" ;
		tile:north_pole = "0.0 90.0" ;
		tile:projection = "cube_gnomonic" ;
		tile:discretization = "logically_rectangular" ;
		tile:conformal = "FALSE" ;
	double x(nyp, nxp) ;
		x:standard_name = "geographic_longitude" ;
		x:units = "degree_east" ;
	double y(nyp, nxp) ;
		y:standard_name = "geographic_latitude" ;
		y:units = "degree_north" ;
	double dx(nyp, nx) ;
		dx:standard_name = "grid_edge_x_distance" ;
		dx:units = "meters" ;
	double dy(ny, nxp) ;
		dy:standard_name = "grid_edge_y_distance" ;
		dy:units = "meters" ;
	double area(ny, nx) ;
		area:standard_name = "grid_cell_area" ;
		area:units = "m2" ;
	double angle_dx(nyp, nxp) ;
		angle_dx:standard_name = "grid_vertex_x_angle_WRT_geographic_east" ;
		angle_dx:units = "degrees_east" ;
	double angle_dy(nyp, nxp) ;
		angle_dy:standard_name = "grid_vertex_y_angle_WRT_geographic_north" ;
		angle_dy:units = "degrees_north" ;
	char arcx(string) ;
		arcx:standard_name = "grid_edge_x_arc_type" ;
		arcx:north_pole = "0.0 90.0" ;

// global attributes:
		:grid_version = "0.2" ;
		:history = "/home/z1l/bin/tools_20091028/make_hgrid --grid_type gnomonic_ed --nlon 96" ;
}
\end{verbatim}

The tile file not only defines the coordinates at all staggers, it also has a complete specification of
distances, angles, and areas.  In ESMF, we only use the {\tt geographic\_longitude} and {\tt geographic\_latitude}
variables and its subsets on the center and corner staggers.  ESMF currently supports the Mosaic containing tiles of the same size.  
A tile can be square or rectangular.  For a cubed sphere grid, each tile is a square, i.e. the x and y
dimensions are the same.   

\subsection{Regrid Weight File Format}\label{sec:weightfileformat}

A regrid weight file is a NetCDF format file containing the information necessary to perform 
a regridding between two grids. It also optionally contains information about the grids used to compute
the regridding. This information is provided to allow applications (e.g. {\tt ESMF\_RegridWeightGenCheck}) to
independently compute the accuracy of the regridding weights. In some cases, {\tt ESMF\_RegridWeightGen} doesn't
output the full grid information (e.g. when it's costly to compute, or when the current grid format doesn't 
support the type of grids used to generate the weights). In that case, the weight file can still be used
for regridding, but applications which depend on the grid information may not work. 

The following is the header of a sample regridding weight file that describes a bilinear regridding 
from a logically rectangular 2D grid to a triangular unstructured grid:

\begin{verbatim}
netcdf t42mpas-bilinear {
dimensions:
        n_a = 8192 ;
        n_b = 20480 ;
        n_s = 42456 ;
        nv_a = 4 ;
        nv_b = 3 ;
        num_wgts = 1 ;
        src_grid_rank = 2 ;
        dst_grid_rank = 1 ;
variables:
        int src_grid_dims(src_grid_rank) ;
        int dst_grid_dims(dst_grid_rank) ;
        double yc_a(n_a) ;
               yc_a:units = "degrees" ;
        double yc_b(n_b) ;
               yc_b:units = "radians" ;
        double xc_a(n_a) ;
               xc_a:units = "degrees" ;
        double xc_b(n_b) ;
               xc_b:units = "radians" ;
        double yv_a(n_a, nv_a) ;
               yv_a:units = "degrees" ;
        double xv_a(n_a, nv_a) ;
               xv_a:units = "degrees" ;
        double yv_b(n_b, nv_b) ;
               yv_b:units = "radians" ;
        double xv_b(n_b, nv_b) ;
               xv_b:units = "radians" ;
        int mask_a(n_a) ;
               mask_a:units = "unitless" ;
        int mask_b(n_b) ;
               mask_b:units = "unitless" ;
        double area_a(n_a) ;
               area_a:units = "square radians" ;
        double area_b(n_b) ;
               area_b:units = "square radians" ;
        double frac_a(n_a) ;
               frac_a:units = "unitless" ;
        double frac_b(n_b) ;
               frac_b:units = "unitless" ;
        int col(n_s) ;
        int row(n_s) ;
        double S(n_s) ;
 
// global attributes:
        :title = "ESMF Offline Regridding Weight Generator" ;
        :normalization = "destarea" ;
        :map_method = "Bilinear remapping" ;
        :ESMF_regrid_method = "Bilinear" ;
        :conventions = "NCAR-CSM" ;
        :domain_a = "T42_grid.nc" ;
        :domain_b = "grid-dual.nc" ;
        :grid_file_src = "T42_grid.nc" ;
        :grid_file_dst = "grid-dual.nc" ;
        :ESMF_version = "ESMF_8_2_0_beta_snapshot_05-3-g2193fa3f8a" ;
}
\end{verbatim}

 The weight file contains four types of information: a description of the source grid, a description of the destination grid, the output of the regrid weight calculation, and global attributes describing the weight file. 

\subsubsection{Source Grid Description}

The variables describing the source grid in the weight file end with the suffix "\_a". To be consistent with the original use of this weight file format 
the grid information is written to the file such that the location being regridded is always the cell center. This means that the grid structure described here may not be identical to that in the source grid file. The full set of these variables may not always be present in the weight file. The following is an 
explanation of each variable:
\begin{description}
  \itemsep0em
  \item[n\_a] The number of source cells. 
  \item[nv\_a] The maximum number of corners (i.e. vertices)  around a source cell. If a cell has less than the maximum number of corners, then the remaining corner coordinates are repeats of the last valid corner's coordinates.    
  \item[xc\_a] The longitude coordinates of the centers of each source cell.  
  \item[yc\_a] The latitude coordinates of the centers of each source cell.  
  \item[xv\_a] The longitude coordinates of the corners of each source cell.  
  \item[yv\_a] The latitude coordinates of the corners of each source cell.  
  \item[mask\_a] The mask for each source cell. A value of 0, indicates that the cell is masked. 
  \item[area\_a] The area of each source cell. This quantity is either from the source grid file or calculated by {\tt ESMF\_RegridWeightGen}. When a non-conservative regridding method (e.g. bilinear) is used, the area is set to 0.0. 
  \item[src\_grid\_rank] The number of dimensions of the source grid. Currently this can only be 1 or 2. Where 1 indicates an unstructured grid and 2 indicates a 2D logically rectangular grid.
  \item[src\_grid\_dims] The number of cells along each dimension of the source grid. For unstructured grids this is equal to the number of cells in the grid. 
\end{description}

\subsubsection{Destination Grid Description}

The variables describing the destination grid in the weight file end with the suffix "\_b". To be consistent with the original use of this weight file format 
the grid information is written to the file such that the location being regridded is always the cell center. This means that the grid structure described here may not be identical to that in the destination grid file. The full set of these variables may not always be present in the weight file. The following is an 
explanation of each variable:
\begin{description}
  \itemsep0em
  \item[n\_b] The number of destination cells. 
  \item[nv\_b] The maximum number of corners (i.e. vertices)  around a destination cell. If a cell has less than the maximum number of corners, then the remaining corner coordinates are repeats of the last valid corner's coordinates.    
  \item[xc\_b] The longitude coordinates of the centers of each destination cell.  
  \item[yc\_b] The latitude coordinates of the centers of each destination cell.  
  \item[xv\_b] The longitude coordinates of the corners of each destination cell.  
  \item[yv\_b] The latitude coordinates of the corners of each destination cell.  
  \item[mask\_b] The mask for each destination cell. A value of 0, indicates that the cell is masked. 
  \item[area\_b] The area of each destination cell. This quantity is either from the destination grid file or calculated by {\tt ESMF\_RegridWeightGen}. When a non-conservative regridding method (e.g. bilinear) is used, the area is set to 0.0. 
 \item[dst\_grid\_rank] The number of dimensions of the destination grid. Currently this can only be 1 or 2. Where 1 indicates an unstructured grid and 2 indicates a 2D logically rectangular grid.
 \item[dst\_grid\_dims] The number of cells along each dimension of the destination grid. For unstructured grids this is equal to the number of cells in the grid. 
\end{description}

\subsubsection{Regrid Calculation Output}\label{regridoutput}

The following is an explanation of the variables containing the output of the regridding calculation:
\begin{description}
  \itemsep0em
  \item[n\_s] The number of entries in the regridding matrix. 
  \item[col] The position in the source grid for each entry in the regridding matrix. 
  \item[row] The position in the destination grid for each entry in the weight matrix. 
  \item[S] The weight for each entry in the regridding matrix. 
  \item[frac\_a] When a conservative regridding method is used, this contains the fraction of each source cell that participated in the regridding. When a non-conservative regridding method is used, this array is set to 0.0.
  \item[frac\_b] When a conservative regridding method is used, this contains the fraction of each destination cell that participated in the regridding. When a non-conservative regridding method is used, this array is set to 1.0 where the point participated in the regridding (i.e. was within the unmasked source grid), and 0.0 otherwise. 
\end{description}

The following code shows how to apply the weights in the weight file to interpolate a source field ({\tt src\_field}) defined 
over the source grid to a destination field ({\tt dst\_field}) defined over the destination grid. The variables {\tt n\_s}, {\tt n\_b}, 
{\tt row}, {\tt col}, and {\tt S} are from the weight file. 

\begin{verbatim}
 ! Initialize destination field to 0.0
 do i=1, n_b
   dst_field(i)=0.0
 enddo

 ! Apply weights
 do i=1, n_s
   dst_field(row(i))=dst_field(row(i))+S(i)*src_field(col(i))
 enddo
\end{verbatim}

If the first-order conservative interpolation method is specified ("-m conserve") then the destination field may need to be adjusted by the destination fraction ({\tt frac\_b}). 
This should be done if the normalization type is "dstarea" and if the destination grid extends outside the unmasked source grid. If it isn't known if the destination extends outside the source, then it doesn't hurt to apply the destination fraction. (If it doesn't extend outside, then the fraction will be 1.0 everywhere anyway.) 
The following code shows how to adjust an already interpolated destination field ({\tt dst\_field}) by the destination fraction. The variables {\tt n\_b}, and {\tt frac\_b} are from the weight file:

\begin{verbatim}
 ! Adjust destination field by fraction
 do i=1, n_b
   if (frac_b(i) .ne. 0.0) then
      dst_field(i)=dst_field(i)/frac_b(i)
   endif
 enddo
\end{verbatim}


\subsubsection{Weight File Description Attributes}

The following is an explanation of the global attributes describing the weight file:
\begin{description}
  \itemsep0em
  \item[title] Always set to  "ESMF Offline Regridding Weight Generator" when generated by {\tt ESMF\_RegridWeightGen}.
  \item[normalization] The normalization type used to compute conservative regridding weights. The options for this are described in section~\ref{sec:rwg_conserve} which contains a description of the conservative regridding method.  
  \item[map\_method] An indication of the mapping method which is constrained by the original use of this format. In some cases the method specified here will differ from the actual regridding method used, for example weights generated with the "patch" method will have this attribute set to "Bilinear remapping". 
  \item[ESMF\_regrid\_method] The ESMF regridding method used to generate the weight file. 
  \item[conventions] The set of conventions that the weight file follows. Currently only "NCAR-CSM" is supported.
  \item[domain\_a] The source grid file name. 
  \item[domain\_b] The destination grid file name. 
  \item[grid\_file\_src] The source grid file name. 
  \item[grid\_file\_dst] The destination grid file name. 
  \item[ESMF\_version] The version of ESMF used to generate the weight file.
\end{description}

\subsubsection{Weight Only Weight File}

In the current ESMF distribution, a new simplified weight file option {\tt --weight\_only} is added to {\tt ESMF\_RegridWeightGen}.  
The simple weight file contains only a subset of the Regrid Calculation Output defined in ~\ref{regridoutput}, i.e. 
the weights {\tt S}, the source grid indices {\tt col} and destination grid indices {\tt row}.  The dimension of these three variables is {\tt n\_s}.

\subsection{ESMF\_RegridWeightGenCheck}\label{sec:regridweightgencheck}

The ESMF\_RegridWeightGen application is used in the
\htmladdnormallink{ESMF\_RegridWeightGenCheck external demo}{http://www.earthsystemmodeling.org/users/code_examples/external_demos/external_demos.shtml} to generate interpolation weights.  These weights are then tested by using them for a regridding operation and then comparing them against an analytic function on the destination grid.  This external demo is also used to regression test ESMF regridding, and it is run nightly on over 150 combinations of structured and unstructured, regional and global grids, and regridding methods.
